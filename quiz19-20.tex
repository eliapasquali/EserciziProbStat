\documentclass{report}
\usepackage[utf8]{inputenc}
\usepackage{eurosym}

\author{Elia Pasquali}

\begin{document}
\begin{center}
    {\Large\textbf{Premessa}}
\end{center}
Questa è la riscrittura di una raccolta di esercizi tratti dai quiz online degli ultimi anni.\\
Non è assicurata la correttezza delle risposte quindi tutte le correzioni e aggiunte sono ben accette.\\

\textbf{Esercizio 1}\\
Siano $A, B$ eventi \textit{indipendenti} tali che $P(A^c) = 0.1$ e $P(B) = 0.7$. Qual è la probabilità dell'evento $A \cap B^c$?
\begin{itemize}
    \item $0.63$
    \item $0.03$
    \item Non la posso calcolare
    \item $0.27$ (Risposta Corretta)
    \item $0.07$
\end{itemize}

\textbf{Esercizio 2}\\
Siano $A,B,C$ eventi \textit{indipendenti} tali che $P(A) = P(B) = P(C) = \frac{1}{2}$. Qual è la probabilità dell'evento $(A \cap B) \cap C$?
\begin{itemize}
    \item Non la posso calcolare
    \item $\frac{1}{8}$ (Risposta Corretta)
    \item $\frac{7}{8}$
    \item $\frac{3}{8}$ 
    \item $\frac{5}{8}$
\end{itemize}

\textbf{Esercizio 3}\\
Lanciamo un dado. Consideriamo i seguenti eventi: $E_\ge = $"esce un punteggio maggiore o uguale a 4", $E_i =$ "esce il punteggio $i$" (con $i = 1,..., 6)$ e $E =$ "esce un punteggio divisibile per 3". Quale delle seguenti affermazioni è \textit{falsa}?
\begin{itemize}
    \item Gli eventi $E_4$ ed $E$ sono incompatibili
    \item Gli eventi $E_1$, $E_2$, $E$ ed $E_\ge$ non sono una partizione
    \item Gli eventi $E_i$ con $i = 1,...,6$ sono una partizione
    \item Nessuna delle altre risposte  (Risposta corretta)
    \item Gli eventi $E_1$, $E_2$, $E_3$ ed $E_\ge$ non sono una partizione
\end{itemize}

\textbf{Esercizio 4}\\
Sia $A,B$ eventi tali che $P(A \cup B) = 0.4$. Qual è la probabilità dell'evento $A^c \cap B^c$?
\begin{itemize}
    \item Non la posso calcolare
    \item $0.3$
    \item $0.7$
    \item $0.4$
    \item $0.6$ (Risposta Corretta)
\end{itemize}

\textbf{Esercizio 5}\\
Holly si allena tirando ai rigori. Ad ogni tiro, indipendentemente dagli altri, ha la probabilità 0.8 di segnare. Qual è la probabilità che segni tutti i prossimi 3 tiri?
\begin{itemize}
    \item $0.512$ (Risposta Corretta)
    \item $0.008$
    \item $0.6$
    \item Nessuna delle altre risposte
    \item $1$
\end{itemize}

\textbf{Esercizio 6}\\
In una classe di 8 alunni, 3 non hanno fatto i compiti. Se l'insegnante ne sceglie 2 a caso, qual è la probabilità che entrambi abbiano svolto i compiti?
\begin{itemize}
    \item $\frac{10}{14}$
    \item $\frac{5}{14}$ (Risposta Corretta)
    \item Nessuna delle altre risposte
    \item $\frac{5}{28}$
    \item $\frac{5}{8}$
\end{itemize}

\textbf{Esercizio 7}\\
La mia amica Martina adora lo yoga e segue un corso avanzato tre volte alla settimana. Il 60\% dei partecipanti a questo corso sono donne e la probabilità che una donna abbia un abbigliamento sportivo sui toni del viola è del 10\%. Sapendo che si è scelta una donna della classe, qual è la probabilità che indossi qualcosa di viola?
\begin{itemize}
    \item $\frac{3}{5}$
    \item Nessuna delle altre risposte
    \item $\frac{1}{6}$
    \item $\frac{7}{10}$
    \item $\frac{1}{10}$ (Risposta Corretta)
\end{itemize}

\textbf{Esercizio 8}\\
La ruota della roulette ha 18 sezioni rosse, 18 nere e 1 verde. Se si punta \euro 1 sul colore rosso (risp. nero) e la pallina si ferma in una sezione rossa (risp. nera), allora si vince €1; altrimenti si perde la putnata. Sia $X$ il guadagno di un giocatore che fa questo tipo di scommessa. La distribuzione di $X$ è la seguente:\\
$px(-1) = \frac{19}{37}$ (perdita) , $px(1) = \frac{18}{37}$ (vincita).\\
Si calcolano: $E(X)\approx -0.03$ e $Var(X)\approx 1$. Supponiamo che un giocatore decida di puntare \euro 5, invece che \euro 1, cosicchè potrebbe vincere o perdere \euro 5. Sia $Z$ la vincita derivante da una puntata di \euro 5 sul nero. Quanto vale (Circa) il valore medio di $Z$?
\begin{itemize}
    \item $\frac{3}{5}$
    \item Nessuna delle altre risposte (Risposta Corretta)
    \item $\frac{1}{6}$
    \item $\frac{7}{10}$
    \item $\frac{1}{10}$
\end{itemize}

\textbf{Esercizio 9}\\
Estraggo una carta da un mazzo di carte da Poker (52 carte). Qual è la probabilità di sceglie picche sapendo che la carta scelta è nera?
\begin{itemize}
    \item $\frac{1}{26}$
    \item $\frac{1}{13}$
    \item $\frac{1}{4}$
    \item $\frac{1}{2}$ (Risposta Corretta)
    \item Nessuna delle altre risposte
\end{itemize}

\textbf{Esercizio 10}\\
Sia $X$ una variabile aleatoria discreta tale che $P(X=0)=P(X=1)=\frac{1}{4}$ e $P(X=-2)=\frac{1}{2}$. Definiamo $Z=2X^3-X^2$. Allora $P(Z=1)$ vale
\begin{itemize}
    \item $0$
    \item Nessuna delle altre risposte
    \item $\frac{1}{2}$
    \item $\frac{1}{4}$ (Risposta Corretta)
    \item $\frac{1}{2}$
\end{itemize}

\textbf{Esercizio 11}\\
Un barattolo di caramelle contiene 6 gelatine rosse, 4 gelatine verdi e 4 gelatine blu. Ne prendo due a caso, qual è la probabilità che siano entrambe blu?
\begin{itemize}
    \item $\frac{3}{91}$
    \item $\frac{8}{14}$
    \item Nessuna delle altre risposte
    \item $\frac{12}{91}$
    \item $\frac{6}{91}$ (Risposta Corretta)
\end{itemize}

\textbf{Esercizio 12}\\
Sia $X$ una variabile aleatoria discreta tale che $P(X=0)=\frac{1}{4}, P(X=1)=\frac{1}{4}$ e $P(X=2)=\frac{1}{2}$. Allora vale
\begin{itemize}
    \item $\frac{9}{4}$ (Risposta Corretta)
    \item $\frac{9}{16}$
    \item $1$
    \item $\frac{11}{16}$
    \item Nessuna delle altre risposte
\end{itemize}

\textbf{Esercizio 13}\\
Lancio due dadi. Qual è la probabilità che la somma dei due punteggia sia maggiore di 3?
\begin{itemize}
    \item Nessuna delle altre risposte
    \item $\frac{1}{12}$
    \item $\frac{35}{36}$
    \item $\frac{1}{18}$
    \item $\frac{11}{12}$ (Risposta Corretta)
\end{itemize}

\textbf{Esercizio 14}\\
Un'urna contiene 2 palline rosse, 3 palline gialle e 2 palline blu. Pesco due palline \textit{con reinserimento}. Qual è la probabilità che nessuna pallina sia blu?
\begin{itemize}
    \item $\frac{1}{2}$
    \item $\frac{10}{21}$
    \item $\frac{20}{21}$
    \item $\frac{25}{49}$ (Risposta Corretta)
    \item Nessuna delle altre risposte
\end{itemize}

\textbf{Esercizio 15} \underline{Nota}: alcune parti del testo erano indecifrabili\\
Se si analizzano dei dati ... ci si accorge che la prima cifra di questi dati non è con uguale probabilità una delle cifre tra 1 e 9. La prima cifra più comune è 1, seguita da 2 e così via, in ordine, fino a 9 che è la prima cifra meno frequente. Questo fenomeno è noto con il nome \textit{legge di Benford}. Sia $D$ la variabile aleatoria che mi da il valore di un dato numerico da questa legge, la ... di $D$ è la seguente:\\
$p_D(1)=0.301, p_D(2)=0.176, p_D(3)=0.125, p_D(4)=0.097, p_D(5)=0.079, p_D(6)=0.067, p_D(7)=0.058, p_D(8)=0.051, p_D(9)=0.046$\\
Quanto vale P(...)?
\begin{itemize}
    \item $0.222$
    \item Nessuna delle altre risposte
    \item $0.067$
    \item $0.155$
    \item $0.051$
\end{itemize}

\textbf{Esercizio 16} \underline{Nota}: alcune parti del testo indecifrabili\\
Durante un sondaggio, ad un gruppo di ragazzi è stato chiesto quale superpotere avrebbero voluto avere. Le risposte sono sintetizzate nella seguente tabella
\begin{center}
    \begin{tabular}{c|cc|c}
        & Maschi & Femmine & Totale \\
        Saper volare & 30 & 10 & 40 \\
        Invisibilità & 12 & 32 & 44 \\
        A... & 10 & 6 & 16 \\
        Totale & 52 & 48 & 100 \\
    \end{tabular}
\end{center}
Si scelga a caso un ragazzo di questo gruppo e si considerino gli eventi $A$="Il ragazzo scelto è maschio" e $B$="il ragazzo scelto vorrebbe saper volare". Allora $P(A|B)$ vale
\begin{itemize}
    \item $\frac{3}{10}$
    \item Nessuna delle altre risposte
    \item $\frac{15}{26}$
    \item $\frac{3}{4}$ (Risposta Corretta)
    \item $\frac{10}{13}$
\end{itemize}

\textbf{Esercizio 17}\\
Benji si allena parando i rigori. Ad ogni tiro, indipendentemente dall'altro, ha una probabilità di 0.8 di parare. Qual è la probabilità che pari tutti i prossimi 3 tiri?
\begin{itemize}
    \item $0.512$ (Risposta Corretta)
    \item Nessuna delle altre risposte
    \item $0.6$
    \item $1$
    \item $0.008$
\end{itemize}

\textbf{Esercizio 18}\\
Siano $A, B$ eventi tali che $P(A)=0.2, P(B)=0.5$ e $P(A|B)=0.4$. Quanto vale la probabilità dell'evento $A\cup B$?
\begin{itemize}
    \item $0.7$
    \item $0.78$
    \item Nessuna delle altre risposte (Risposta Corretta)
    \item $0.62$
    \item $0.24$
\end{itemize}

\textbf{Esercizio 19}\\
Lancio due dadi. Qual è la probabilità che la somma dei due punteggia sia 4?
\begin{itemize}
    \item $\frac{1}{18}$
    \item $\frac{1}{6}$
    \item $\frac{1}{36}$
    \item $\frac{1}{12}$ (Risposta Corretta)
    \item Nessuna delle altre risposte
\end{itemize}

\textbf{Esercizio 20} \underline{Nota}: alcune parti del testo indecifrabili\\
Mia mamma non era una gran cuoca ... 5 formati di pasta (spaghetti, penne, farfalle, fettuccine, fusilli) e 3 condimenti (pomodoro, ragù, carbonara), che abbinava a caso. Qual era la probabilità di non mangiare fusilli al ragù?
\begin{itemize}
    \item Nessuna delle altre risposte
    \item $\frac{2}{3}$
    \item $\frac{4}{5}$
    \item $\frac{14}{15}$ (Risposta Corretta)
    \item $\frac{1}{15}$
\end{itemize}

\textbf{Esercizio 21} \underline{Nota}: alcune parti del testo indecifrabili\\
Le lettere della parola STATISTICA vengono ... e mescolate. Un ... viene pescato a caso. Qual è la probabilità di pescare una consonante?
\begin{itemize}
    \item $\frac{1}{5}$
    \item $\frac{5}{11}$
    \item $\frac{2}{5}$
    \item Nessuna delle altre risposte (Risposta Corretta)
    \item $\frac{6}{11}$
\end{itemize}

\textbf{Esercizio 22} \underline{Nota}: alcune parti del testo indecifrabili\\
Da un mazzo di carte da Poker (52 carte) vengono pescate tre carte \textit{senza reinserimento}. Si considerino gli eventi $A$="le prime due carte sono entrambe di cuori" e $B$="la prima e la terza carta non sono dello stesso seme". Allora
\begin{itemize}
    \item $A\cap B^c=$"le tre carte sono di cuori" (Risposta Corretta)
    \item $A\cap B$ è l'evento impossibile 
    \item Nessuna delle altre risposte
    \item $A\cup B=$"le prime due carte sono di cuori e la terza non è di cuori"
    \item $A^c\cap B=$"la seconda e la terza carta non sono entrambe di cuori"
\end{itemize}

\textbf{Esercizio 23} \underline{Nota}: alcune parti del testo indecifrabili\\
Siano $A, B$ eventi tali che $P(A)=0.2, P(B)=0.3$. Qual è la probabilità dell'evento $A\cup B$?
\begin{itemize}
    \item $0.5$
    \item $0.66$
    \item $0.44$
    \item $0.6$
    \item Nessuna delle altre risposte  (Risposta Corretta)
\end{itemize}

\textbf{Esercizio 24} \underline{Nota}: alcune parti del testo indecifrabili\\
Impresa d'appalti e distribuzione $X$\\

\textbf{Esercizio 25}\\
Siano $A, B, C$ eventi \textit{indipendenti} tali che $P(A)=P(B)=P(C)=\frac{1}{2}$. Qual è la probabilità dell'evento $(A\cap B)\cup C$?
\begin{itemize}
    \item Non la posso calcolare
    \item $7/8$
    \item $5/8$ (Risposta corretta)
    \item $1/8$
    \item $3/8$
\end{itemize}

\textbf{Esercizio 26}\\
Sia $X$ una variabile aleatoria discreta tale che $P(X=0)=P(X=1)=\frac{1}{4}$ e $P(X=-2)=\frac{1}{2}$. Definiamo $Y=X-X^2$. Allora l'alfabeto di $Y$ è
\begin{itemize}
    \item ${0, -2, -6}$ 
    \item ${0, -2}$
    \item ${0, -2, 2}$
    \item ${0, -6}$
    \item Nessuna delle altre risposte
\end{itemize}

\textbf{Esercizio 27}\\
Lanciamo un dado. Consideriamo i seguenti eventi: $E_> = $"esce un punteggio maggiore di 4", $E_i =$ "esce il punteggio $i$" (con $i = 1,..., 6)$ e $E =$ "esce un punteggio divisibile per 3". Quale delle seguenti affermazioni è \textit{falsa}?
\begin{itemize}
    \item Gli eventi $E_5$ ed $E$ sono incompatibili
    \item Nessuna delle altre risposte (Risposta corretta)
    \item Gli eventi $E_1$, $E_2$, $E_3$ ed $E_>$ non sono una partizione
    \item Gli eventi $E_i$ con $i = 1,...,6$ sono una partizione
    \item Gli eventi $E_4$, $E_2$, $E_1$, $E$ ed $E_>$ non sono una partizione
\end{itemize}

\end{document}
