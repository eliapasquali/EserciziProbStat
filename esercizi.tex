\documentclass{report}
\usepackage[utf8]{inputenc}
\usepackage{eurosym}
\usepackage{multicol}
\usepackage{multirow}
\usepackage{graphicx}
\usepackage{amsmath}

\author{Elia Pasquali}

\begin{document}

\begin{center}
    {\Large\textbf{Esercizi di Probabilità e Statistica}}
\end{center}
Questa raccolta di esercizi è tratta dai vari quiz online degli ultimi anni. 

\section*{Anno 19/20}
Gli esercizi di questa sezione sono stati ricontrollati e dovrebbero essere corretti. Le fonti di alcuni erano illeggibili e quindi la riscrittura non è completa.\\

\textbf{Esercizio 1}
Siano $A, B$ eventi \textit{indipendenti} tali che $P(A^c) = 0.1$ e $P(B) = 0.7$. Qual è la probabilità dell'evento $A \cap B^c$?
\begin{itemize}
    \item $0.63$
    \item $0.03$
    \item Non la posso calcolare
    \item $0.27$ (Risposta Corretta)
    \item $0.07$
\end{itemize}

\textbf{Esercizio 2}
Siano $A,B,C$ eventi \textit{indipendenti} tali che $P(A) = P(B) = P(C) = \frac{1}{2}$. Qual è la probabilità dell'evento $(A \cap B) \cap C$?
\begin{itemize}
    \item Non la posso calcolare
    \item $\frac{1}{8}$ (Risposta Corretta)
    \item $\frac{7}{8}$
    \item $\frac{3}{8}$ 
    \item $\frac{5}{8}$
\end{itemize}

\textbf{Esercizio 3}
Lanciamo un dado. Consideriamo i seguenti eventi: $E_\ge = $"esce un punteggio maggiore o uguale a 4", $E_i =$ "esce il punteggio $i$" (con $i = 1,..., 6)$ e $E =$ "esce un punteggio divisibile per 3". Quale delle seguenti affermazioni è \textit{falsa}?
\begin{itemize}
    \item Gli eventi $E_4$ ed $E$ sono incompatibili
    \item Gli eventi $E_1$, $E_2$, $E$ ed $E_\ge$ non sono una partizione
    \item Gli eventi $E_i$ con $i = 1,...,6$ sono una partizione
    \item Nessuna delle altre risposte  (Risposta corretta)
    \item Gli eventi $E_1$, $E_2$, $E_3$ ed $E_\ge$ non sono una partizione
\end{itemize}

\textbf{Esercizio 4}
Sia $A,B$ eventi tali che $P(A \cup B) = 0.4$. Qual è la probabilità dell'evento $A^c \cap B^c$?
\begin{itemize}
    \item Non la posso calcolare
    \item $0.3$
    \item $0.7$
    \item $0.4$
    \item $0.6$ (Risposta Corretta)
\end{itemize}

\textbf{Esercizio 5}
Holly si allena tirando ai rigori. Ad ogni tiro, indipendentemente dagli altri, ha la probabilità 0.8 di segnare. Qual è la probabilità che segni tutti i prossimi 3 tiri?
\begin{itemize}
    \item $0.512$ (Risposta Corretta)
    \item $0.008$
    \item $0.6$
    \item Nessuna delle altre risposte
    \item $1$
\end{itemize}

\textbf{Esercizio 6}
In una classe di 8 alunni, 3 non hanno fatto i compiti. Se l'insegnante ne sceglie 2 a caso, qual è la probabilità che entrambi abbiano svolto i compiti?
\begin{itemize}
    \item $\frac{10}{14}$
    \item $\frac{5}{14}$ (Risposta Corretta)
    \item Nessuna delle altre risposte
    \item $\frac{5}{28}$
    \item $\frac{5}{8}$
\end{itemize}

\textbf{Esercizio 7}
La mia amica Martina adora lo yoga e segue un corso avanzato tre volte alla settimana. Il 60\% dei partecipanti a questo corso sono donne e la probabilità che una donna abbia un abbigliamento sportivo sui toni del viola è del 10\%. Sapendo che si è scelta una donna della classe, qual è la probabilità che indossi qualcosa di viola?
\begin{itemize}
    \item $\frac{3}{5}$
    \item Nessuna delle altre risposte
    \item $\frac{1}{6}$
    \item $\frac{7}{10}$
    \item $\frac{1}{10}$ (Risposta Corretta)
\end{itemize}

\textbf{Esercizio 8}
La ruota della roulette ha 18 sezioni rosse, 18 nere e 1 verde. Se si punta \euro 1 sul colore rosso (risp. nero) e la pallina si ferma in una sezione rossa (risp. nera), allora si vince €1; altrimenti si perde la puntata. Sia $X$ il guadagno di un giocatore che fa questo tipo di scommessa. La distribuzione di $X$ è la seguente:\\
$px(-1) = \frac{19}{37}$ (perdita) , $px(1) = \frac{18}{37}$ (vincita).\\
Si calcolano: $E(X)\approx -0.03$ e $Var(X)\approx 1$. Supponiamo che un giocatore decida di puntare \euro 5, invece che \euro 1, cosicchè potrebbe vincere o perdere \euro 5. Sia $Z$ la vincita derivante da una puntata di \euro 5 sul nero. Quanto vale (Circa) il valore medio di $Z$?
\begin{itemize}
    \item $\frac{3}{5}$
    \item Nessuna delle altre risposte (Risposta Corretta)
    \item $\frac{1}{6}$
    \item $\frac{7}{10}$
    \item $\frac{1}{10}$
\end{itemize}

\textbf{Esercizio 9}
Estraggo una carta da un mazzo di carte da Poker (52 carte). Qual è la probabilità di sceglie picche sapendo che la carta scelta è nera?
\begin{itemize}
    \item $\frac{1}{26}$
    \item $\frac{1}{13}$
    \item $\frac{1}{4}$
    \item $\frac{1}{2}$ (Risposta Corretta)
    \item Nessuna delle altre risposte
\end{itemize}

\textbf{Esercizio 10}
Sia $X$ una variabile aleatoria discreta tale che $P(X=0)=P(X=1)=\frac{1}{4}$ e $P(X=-2)=\frac{1}{2}$. Definiamo $Z=2X^3-X^2$. Allora $P(Z=1)$ vale
\begin{itemize}
    \item $0$
    \item Nessuna delle altre risposte
    \item $\frac{1}{2}$
    \item $\frac{1}{4}$ (Risposta Corretta)
    \item $\frac{1}{2}$
\end{itemize}

\textbf{Esercizio 11}
Un barattolo di caramelle contiene 6 gelatine rosse, 4 gelatine verdi e 4 gelatine blu. Ne prendo due a caso, qual è la probabilità che siano entrambe blu?
\begin{itemize}
    \item $\frac{3}{91}$
    \item $\frac{8}{14}$
    \item Nessuna delle altre risposte
    \item $\frac{12}{91}$
    \item $\frac{6}{91}$ (Risposta Corretta)
\end{itemize}

\textbf{Esercizio 12}
Sia $X$ una variabile aleatoria discreta tale che $P(X=0)=\frac{1}{4}, P(X=1)=\frac{1}{4}$ e $P(X=2)=\frac{1}{2}$. Allora vale
\begin{itemize}
    \item $\frac{9}{4}$ (Risposta Corretta)
    \item $\frac{9}{16}$
    \item $1$
    \item $\frac{11}{16}$
    \item Nessuna delle altre risposte
\end{itemize}

\textbf{Esercizio 13}
Lancio due dadi. Qual è la probabilità che la somma dei due punteggia sia maggiore di 3?
\begin{itemize}
    \item Nessuna delle altre risposte
    \item $\frac{1}{12}$
    \item $\frac{35}{36}$
    \item $\frac{1}{18}$
    \item $\frac{11}{12}$ (Risposta Corretta)
\end{itemize}

\textbf{Esercizio 14}
Un'urna contiene 2 palline rosse, 3 palline gialle e 2 palline blu. Pesco due palline \textit{con reinserimento}. Qual è la probabilità che nessuna pallina sia blu?
\begin{itemize}
    \item $\frac{1}{2}$
    \item $\frac{10}{21}$
    \item $\frac{20}{21}$
    \item $\frac{25}{49}$ (Risposta Corretta)
    \item Nessuna delle altre risposte
\end{itemize}

\textbf{Esercizio 15} \underline{Nota}: alcune parti del testo erano indecifrabili\\
Se si analizzano dei dati ... ci si accorge che la prima cifra di questi dati non è con uguale probabilità una delle cifre tra 1 e 9. La prima cifra più comune è 1, seguita da 2 e così via, in ordine, fino a 9 che è la prima cifra meno frequente. Questo fenomeno è noto con il nome \textit{legge di Benford}. Sia $D$ la variabile aleatoria che mi da il valore di un dato numerico da questa legge, la ... di $D$ è la seguente:\\
$p_D(1)=0.301, p_D(2)=0.176, p_D(3)=0.125, p_D(4)=0.097, p_D(5)=0.079, p_D(6)=0.067, p_D(7)=0.058, p_D(8)=0.051, p_D(9)=0.046$\\
Quanto vale P(...)?
\begin{itemize}
    \item $0.222$
    \item Nessuna delle altre risposte
    \item $0.067$
    \item $0.155$
    \item $0.051$
\end{itemize}

\textbf{Esercizio 16} \underline{Nota}: alcune parti del testo indecifrabili\\
Durante un sondaggio, ad un gruppo di ragazzi è stato chiesto quale superpotere avrebbero voluto avere. Le risposte sono sintetizzate nella seguente tabella
\begin{center}
    \begin{tabular}{c|cc|c}
        & Maschi & Femmine & Totale \\ \hline
        Saper volare & 30 & 10 & 40 \\
        Invisibilità & 12 & 32 & 44 \\
        Altro & 10 & 6 & 16 \\ \hline
        Totale & 52 & 48 & 100 \\
    \end{tabular}
\end{center}
Si scelga a caso un ragazzo di questo gruppo e si considerino gli eventi $A$="Il ragazzo scelto è maschio" e $B$="il ragazzo scelto vorrebbe saper volare". Allora $P(A|B)$ vale
\begin{itemize}
    \item $\frac{3}{10}$
    \item Nessuna delle altre risposte
    \item $\frac{15}{26}$
    \item $\frac{3}{4}$ (Risposta Corretta)
    \item $\frac{10}{13}$
\end{itemize}

\textbf{Esercizio 17}
Benji si allena parando i rigori. Ad ogni tiro, indipendentemente dall'altro, ha una probabilità di 0.8 di parare. Qual è la probabilità che pari tutti i prossimi 3 tiri?
\begin{itemize}
    \item $0.512$ (Risposta Corretta)
    \item Nessuna delle altre risposte
    \item $0.6$
    \item $1$
    \item $0.008$
\end{itemize}

\textbf{Esercizio 18}
Siano $A, B$ eventi tali che $P(A)=0.2, P(B)=0.5$ e $P(A|B)=0.4$. Quanto vale la probabilità dell'evento $A\cup B$?
\begin{itemize}
    \item $0.7$
    \item $0.78$
    \item Nessuna delle altre risposte (Risposta Corretta)
    \item $0.62$
    \item $0.24$
\end{itemize}

\textbf{Esercizio 19}
Lancio due dadi. Qual è la probabilità che la somma dei due punteggia sia 4?
\begin{itemize}
    \item $\frac{1}{18}$
    \item $\frac{1}{6}$
    \item $\frac{1}{36}$
    \item $\frac{1}{12}$ (Risposta Corretta)
    \item Nessuna delle altre risposte
\end{itemize}

\textbf{Esercizio 20} \underline{Nota}: alcune parti del testo indecifrabili\\
Mia mamma non era una gran cuoca ... 5 formati di pasta (spaghetti, penne, farfalle, fettuccine, fusilli) e 3 condimenti (pomodoro, ragù, carbonara), che abbinava a caso. Qual era la probabilità di non mangiare fusilli al ragù?
\begin{itemize}
    \item Nessuna delle altre risposte
    \item $\frac{2}{3}$
    \item $\frac{4}{5}$
    \item $\frac{14}{15}$ (Risposta Corretta)
    \item $\frac{1}{15}$
\end{itemize}

\textbf{Esercizio 21} \underline{Nota}: alcune parti del testo indecifrabili\\
Le lettere della parola STATISTICA vengono ... e mescolate. Un ... viene pescato a caso. Qual è la probabilità di pescare una consonante?
\begin{itemize}
    \item $\frac{1}{5}$
    \item $\frac{5}{11}$
    \item $\frac{2}{5}$
    \item Nessuna delle altre risposte (Risposta Corretta)
    \item $\frac{6}{11}$
\end{itemize}

\textbf{Esercizio 22} \underline{Nota}: alcune parti del testo indecifrabili\\
Da un mazzo di carte da Poker (52 carte) vengono pescate tre carte \textit{senza reinserimento}. Si considerino gli eventi $A$="le prime due carte sono entrambe di cuori" e $B$="la prima e la terza carta non sono dello stesso seme". Allora
\begin{itemize}
    \item $A\cap B^c=$"le tre carte sono di cuori" (Risposta Corretta)
    \item $A\cap B$ è l'evento impossibile 
    \item Nessuna delle altre risposte
    \item $A\cup B=$"le prime due carte sono di cuori e la terza non è di cuori"
    \item $A^c\cap B=$"la seconda e la terza carta non sono entrambe di cuori"
\end{itemize}

\textbf{Esercizio 23} \underline{Nota}: alcune parti del testo indecifrabili\\
Siano $A, B$ eventi tali che $P(A)=0.2, P(B)=0.3$. Qual è la probabilità dell'evento $A\cup B$?
\begin{itemize}
    \item $0.5$
    \item $0.66$
    \item $0.44$
    \item $0.6$
    \item Nessuna delle altre risposte  (Risposta Corretta)
\end{itemize}

\textbf{Esercizio 24} \underline{Nota}: alcune parti del testo indecifrabili\\
Impresa d'appalti e distribuzione $X$\\

\textbf{Esercizio 25}
Siano $A, B, C$ eventi \textit{indipendenti} tali che $P(A)=P(B)=P(C)=\frac{1}{2}$. Qual è la probabilità dell'evento $(A\cap B)\cup C$?
\begin{itemize}
    \item Non la posso calcolare
    \item $7/8$
    \item $5/8$ (Risposta corretta)
    \item $1/8$
    \item $3/8$
\end{itemize}

\textbf{Esercizio 26}
Sia $X$ una variabile aleatoria discreta tale che $P(X=0)=P(X=1)=\frac{1}{4}$ e $P(X=-2)=\frac{1}{2}$. Definiamo $Y=X-X^2$. Allora l'alfabeto di $Y$ è
\begin{itemize}
    \item ${0, -2, -6}$ 
    \item ${0, -2}$
    \item ${0, -2, 2}$
    \item ${0, -6}$
    \item Nessuna delle altre risposte
\end{itemize}

\textbf{Esercizio 27}
Lanciamo un dado. Consideriamo i seguenti eventi: $E_> = $"esce un punteggio maggiore di 4", $E_i =$ "esce il punteggio $i$" (con $i = 1,..., 6)$ e $E =$ "esce un punteggio divisibile per 3". Quale delle seguenti affermazioni è \textit{falsa}?
\begin{itemize}
    \item Gli eventi $E_5$ ed $E$ sono incompatibili
    \item Nessuna delle altre risposte (Risposta corretta)
    \item Gli eventi $E_1$, $E_2$, $E_3$ ed $E_>$ non sono una partizione
    \item Gli eventi $E_i$ con $i = 1,...,6$ sono una partizione
    \item Gli eventi $E_4$, $E_2$, $E_1$, $E$ ed $E_>$ non sono una partizione
\end{itemize}

\section*{Anno 20/21}
Le risposte degli esercizi di questa sezione sono prese direttamente dalle soluzioni dei questionari.

\subsection*{Primo questionario}

\textbf{Argomenti}\\
Spazi campionari, disposizioni e combinazioni, probabilità condizionate, eventi indipendenti\\

\textbf{Esercizio 1}
Siano $A, B$ eventi \textit{indipendenti} tali che $P(A) = 0.2$ e $P(B) = 0.3$. Qual è la probabilità dell'evento $A \cup B$?
\begin{itemize}
    \item Non la posso calcolare
    \item $0.5$
    \item $0.44$ (Risposta Corretta)
    \item $0.66$
    \item $0.6$
\end{itemize}

\textbf{Esercizio 2}
Siano $A,B$ eventi tali che $P(A|B) = P(B) = \frac{1}{2}$. Qual è la probabilità dell'evento $(A^c \cap B^c)$?
\begin{itemize}
    \item $\frac{1}{4}$
    \item $\frac{1}{2}$
    \item $\frac{3}{8}$ 
    \item Non la posso calcolare
    \item $\frac{3}{4}$ (Risposta Corretta)
\end{itemize}

\textbf{Esercizio 3}
Da un mazzo di carte da Poker (52 carte) si estraggono, in successione e senza reinserimento, tre carte. Si considerino gli eventi $A=$"la prima e la terza carta non sono dello stesso seme" e $B=$"le ultime due carte sono entrambe di fiori". Allora
\begin{itemize}
    \item $A\cap B$ è l'evento impossibile
    \item $A^c\cap B=$"le tre carte sono di fiori" (Risposta Corretta)
    \item Nessuna delle altre risposte
    \item $A\cap B^c=$"la prima e la seconda carta non sono entrambe di fiori"
    \item $A\cup B=$"la prima carta non è di fiori e le due ultime sono di fiori"
\end{itemize}

\textbf{Esercizio 4}
Estraggo una carta da un mazzo di carte da Poker (52 carte). Qual è la probabilità di scegliere una carta di quadri sapendo che la carta scelta è rossa?
\begin{itemize}
    \item $\frac{1}{4}$
    \item $\frac{1}{13}$
    \item $\frac{1}{26}$
    \item Nessuna delle altre risposte
    \item $\frac{1}{2}$ (Risposta Corretta)
\end{itemize}

\textbf{Esercizio 5}
Andrea, Stefano, Luca e Olga hanno formato un quartetto, con quattro diversi strumenti. Se ognuno di essi può suonare tutti e quattro gli strumenti, quanti sono i modi possibili per formare il quartetto?
\begin{itemize}
    \item Nessuna delle altre risposte
    \item $18$
    \item $24$ (Risposta Corretta)
    \item $36$
    \item $20$
\end{itemize}

\textbf{Esercizio 6}
Siano $A, B$ eventi tali che $P(A) = 0.2, P(B) = 0.5 e P(B|A) = 0.4$. Quanto vale la probabilità dell'evento $A\cup B$?
\begin{itemize}
    \item $0.24$
    \item $0.7$
    \item $0.62$ (Risposta Corretta)
    \item $0.78$
    \item Nessuna delle altre risposte
\end{itemize}

\textbf{Esercizio 7}
Supponiamo di avere due mazzi di carte da Poker (52 carte) e di pescare una carta da ciascun mazzo. Qual è la probabilità che entrambe le carte siano di quadri?
\begin{itemize}
    \item $\frac{3}{51}$
    \item $\frac{3}{104}$
    \item $\frac{1}{16}$ (Risposta Corretta)
    \item Nessuna delle altre risposte
    \item $\frac{5}{84}$
\end{itemize}

\textbf{Esercizio 8}
Il mio amico Hugo fa il cameriere di sala in un famoso ristorante parigino. Ogni sera serve una decina di tavoli. La probabilità che ad un tavolo chiedano un seggiolone è 0.03. Qual è la probabilità (approssimata) che durante una serata sia necessario almeno un seggiolone?
\begin{itemize}
    \item $0.26$ (Risposta Corretta)
    \item $0.03$
    \item $0.74$
    \item Nessuna delle altre risposte
    \item $0.97$
\end{itemize}

\textbf{Esercizio 9}
Siano $A,B$ eventi tali che $P(A|B^c)=P(B)=\frac{1}{2}$. Qual è la probabilità dell'evento $A^c\cup B$?
\begin{itemize}
    \item $\frac{3}{8}$
    \item $\frac{1}{4}$
    \item $\frac{1}{2}$
    \item Non la posso calcolare
    \item $\frac{3}{4}$ (Risposta Corretta)
\end{itemize}

\textbf{Esercizio 10}
Lanciamo contemporaneamente un dado e una moneta. Qual è la probabilità che il dado dia 6 e la moneta testa?
\begin{itemize}
    \item $\frac{1}{2}$
    \item $\frac{1}{12}$ (Risposta Corretta)
    \item Nessuna delle altre risposte
    \item $\frac{2}{3}$
    \item $\frac{1}{6}$
\end{itemize}

\textbf{Esercizio 11}
In una classe di 8 alunni, 3 hanno fatto i compiti. Se l'insegnante ne sceglie 2 a caso, qual è la probabilità che entrambi abbiano svolto i compiti?
\begin{itemize}
    \item $\frac{3}{8}$
    \item $\frac{3}{56}$
    \item $\frac{3}{28}$ (Risposta Corretta)
    \item $\frac{3}{14}$
    \item Nessuna delle altre risposte
\end{itemize}

\textbf{Esercizio 12}
Siano $A, B$ eventi tali che $P(A)=0.5, P(B)=0.8$ e $P(A\cap B)=0.2$. Quanto vale $P(A|B)$?
\begin{itemize}
    \item $0.4$
    \item $0.65$
    \item $0.25$ (Risposta Corretta)
    \item $0.5$
    \item Nessuna delle altre risposte
\end{itemize}

\textbf{Esercizio 13}
La probabilità che un volo parta in orario è $0.8$, la probabilità che arrivi in orario è $0.4$ e la probabilità che parta e arrivi in orario è $0.2$. Quanto vale la probabilità che un aereo arrivi in orario, dato che è partito in orario?
\begin{itemize}
    \item $0.2$
    \item $0.5$
    \item $0.75$
    \item $0.25$ (Risposta Corretta)
    \item Nessuna delle altre risposte
\end{itemize}

\textbf{Esercizio 14}
Da un mazzo di carte da Poker (52 carte) si estrae una carta. Qual è la probabilità che sia la regina di fiori o il re di quadri?
\begin{itemize}
    \item $\frac{1}{52}$
    \item Nessuna delle altre risposte
    \item $0$
    \item $\frac{1}{26}$ (Risposta Corretta)
    \item $\frac{1}{2704}$
\end{itemize}

\textbf{Esercizio 15}
Estraggo una carta da un mazzo di carte da Poker (52 carte). Considero gli eventi $A=$"estraggo un asso" e $B=$"estraggo una carta di fiori". Quanto vale $P(A|B)$?
\begin{itemize}
    \item $\frac{1}{52}$
    \item $\frac{1}{4}$
    \item $\frac{1}{13}$ (Risposta Corretta)
    \item $\frac{17}{52}$
\end{itemize}

\textbf{Esercizio 16}
Si considerino lo spazio campionario $\Omega =\{Do, Re, Mi, Fa, Sol, La, Si\}$ e gli eventi $A=\{Do, Re, Si\}, B=\{Re, Mi, Fa\}$ e $C=\{La\}$. Quali delle seguenti terne è una terna di eventi a 2 a 2 disgiunti?
\begin{itemize}
    \item $\{A\cup B, C, (A\cup B \cup C)^c\}$ (Risposta Corretta)
    \item $\{A \\ B, B, C^c\}$
    \item $\{A\cap B, B^c, C\}$
    \item Nessuna delle altre risposte
    \item $\{A, B, C\}$
\end{itemize}

\textbf{Esercizio 17}
Le lettere della parola STATISTICA vengono scritte su dei bigliettini che vengono poi messi in un sacchetto e mescolati. Un bigliettino viene quindi pescato a caso. Qual è la probabilità di pescare una vocale?
\begin{itemize}
    \item $\frac{5}{11}$
    \item $\frac{1}{5}$
    \item Nessuna delle altre risposte
    \item $\frac{2}{5}$ (Risposta Corretta)
    \item $\frac{6}{11}$
\end{itemize}

\textbf{Esercizio 18}
Al panificio "La dolce baguette" fanno dei dolci squisiti, a cui è davvero difficile resistere. Ogni persona che entra per prendere il pane, indipendentemente dalle altre, ne è così tentata che con una probabilità $0.85$ aggiungerà un dolce alla sua spesa. Stamattina il panificio ha avuto $7$ clienti. Qual è la probabilità (approssimata) che almeno uno non abbia comperato un dolce?
\begin{itemize}
    \item $0.32$
    \item $0.15$
    \item $0.68$ (Risposta Corretta)
    \item Nessuna delle altre risposte
    \item $\frac{1}{7}$
\end{itemize}

\textbf{Esercizio 19}
Gli studenti di un corso di laurea vengono classificati sulla base di due caratteristiche: il sesso e il tipo di scuola superiore di provenienza. I dati sono riportati nella seguente tabella
\begin{center}
    \begin{tabular}{c|cc|c}
        & LICEO & ALTRA SCUOLA SUPERIORE & Totale \\ \hline
        MASCHI & $47$ & $63$ & $110$ \\
        FEMMINE & $62$ & $51$ & $113$ \\ \hline
        Totale & $109$ & $114$ & $223$ \\
    \end{tabular}
\end{center}
Si scelga a caso uno studente di questo corso di laurea e si considerino gli eventi: $A=$"lo studente scelto è maschio" e $B=$"lo studente scelto proviene da un liceo". Allora $P(A|B)$ vale
\begin{itemize}
    \item $\frac{47}{98}$
    \item $\frac{47}{110}$
    \item Nessuna delle altre risposte
    \item $\frac{47}{223}$
    \item $\frac{47}{109}$ (Risposta Corretta)
\end{itemize}

\textbf{Esercizio 20}
Lanciamo un dado. Consideriamo i seguenti eventi: $E_<=$"esce un punteggio minore di 3", $E_i=$"esce il punteggio $i$" (con $i=1,...6$) e $E=$"esce un punteggio dispari". Quali delle seguenti affermazioni è vera?
\begin{itemize}
    \item Gli eventi $E_2, E_6$ ed $E$ sono una partizione
    \item Nessuna delle altre risposte
    \item Gli eventi $E_i$, con $i=1,...,6$ sono una partizione (Risposta corretta)
    \item Gli eventi $E_4, E_6, E$ ed $E_<$ sono una partizione
    \item Gli eventi $E_4$ ed $E_<$ non sono incompatibili
\end{itemize}

\textbf{Esercizio 21}
Il veliero dell'audace Capitano Kirk è a poche miglia dal galeone del famigerato pirata Gambadilegno. Il capitano ha probabilità $\frac{3}{5}$ di colpire la nave del pirata coi suoi cannoni; mentre il pirata, che ha solo un occhio buono, colpisce la nave del capitano con probabilità $\frac{1}{5}$. Se entrambi danno fuoco alle polveri nello stesso momento, qual è la probabilità che il capitano Kirk colpisca la nave pirata e Gambadilegno invece manchi la nave del capitano?
\begin{itemize}
    \item $\frac{4}{5}$
    \item $\frac{12}{25}$ (Risposta Corretta)
    \item $\frac{2}{5}$
    \item $\frac{4}{25}$
    \item Nessuna delle altre risposte
\end{itemize}

\textbf{Esercizio 22}
Siano $A, B$ eventi tali che $P(A\cap B)=0.6, P(A^c)=0.3$ e $P(B)=0.8$. Qual è la probabilità dell'evento $A\cup B$?
\begin{itemize}
    \item $0.9$ (Risposta Corretta)
    \item $0.5$
    \item Non la posso calcolare
    \item $0.4$
    \item $0.8$
\end{itemize}

\textbf{Esercizio 23}
Benji si allena parando i rigori. Ad ogni tiro, indipendentemente dagli altri, ha la probabilità $0.8$ di parare. Qual è la probabilità che sbagli tutte le prossime $3$ parate?
\begin{itemize}
    \item $0.008$ (Risposta Corretta)
    \item $0.6$
    \item $1$
    \item Nessuna delle altre risposte
    \item $0.512$
\end{itemize}

\textbf{Esercizio 24}
Gli studenti di un corso di laurea vengono classificati sulla base di due caratteristiche: il sesso e il tipo di scuola superiore di provenienza. I dati sono riportati nella seguente tabella
\begin{center}
    \begin{tabular}{c|cc|c}
        & LICEO & ALTRA SCUOLA SUPERIORE & Totale \\ \hline
        MASCHI & $47$ & $63$ & $110$ \\
        FEMMINE & $62$ & $51$ & $113$ \\ \hline
        Totale & $109$ & $114$ & $223$ \\
    \end{tabular}
\end{center}
Si scelga a caso uno studente di questo corso di laurea e si considerino gli eventi: $A=$"lo studente scelto è femmina" e $B=$"lo studente scelto proviene da un liceo". Allora $P(B|A)$ vale
\begin{itemize}
    \item Nessuna delle altre risposte
    \item $\frac{62}{223}$
    \item $\frac{62}{113}$ (Risposta Corretta)
    \item $\frac{62}{125}$
    \item $\frac{62}{109}$
\end{itemize}

\textbf{Esercizio 25}
Stai giocando ad un gioco in cui devi difendere un villaggio da un'invasione di orchi. Ci sono $3$ personaggi (elfo, hobbit, uomo) e $5$ strumenti di difesa (magia, spada, scudo, fionda, ombrello). Se scegli a caso il tuo personaggio e la tua arma, con quale probabilità sarai un umano?
\begin{itemize}
    \item $\frac{1}{5}$
    \item $\frac{1}{15}$
    \item $\frac{8}{15}$
    \item Nessuna delle altre risposte
    \item $\frac{1}{3}$
\end{itemize}

\textbf{Esercizio 26}
Siano $A, B$ eventi tali che $P(A\cap B^c)=0.3$. Qual è la probabilità dell'evento $A^c\cap B$?
\begin{itemize}
    \item $0.3$
    \item Non la posso calcolare (Risposta Corretta)
    \item $0.6$
    \item $0.4$
    \item $0.7$
\end{itemize}

\textbf{Esercizio 27}
La mia amica Martina adora lo yoga e segue un corso avanzato tre volte alla settimana. Il $60\%$ dei partecipanti a questo corso sono donne; il $10\%$ dei partecipanti sono donne con un abbigliamento sui toni del viola. Sapendo che si è scelta una donna della classe, qual è la probabilità che indossi qualcosa di viola?
\begin{itemize}
    \item $\frac{1}{10}$
    \item $\frac{7}{10}$
    \item Nessuna delle altre risposte
    \item $\frac{3}{5}$
    \item $\frac{1}{6}$ (Risposta Corretta)
\end{itemize}

\textbf{Esercizio 28}
Siano $A, B$ eventi \textit{indipendenti} tali che $P(A^c)=0.1$ e $P(B)=0.7$. Qual è la probabilità dell'evento $A^c\cap B$?
\begin{itemize}
    \item Non la posso calcolare
    \item $0.27$
    \item $0.07$ (Risposta Corretta)
    \item $0.03$
    \item $0.63$
\end{itemize}

\textbf{Esercizio 29}
Siano $A, B$ eventi tali che $P(A^c\cup B)=0.4$. Qual è la probabilità dell'evento $A\cup B^c$?
\begin{itemize}
    \item $0.7$
    \item $0.6$
    \item $0.3$
    \item $0.4$
    \item Non la posso calcolare (Risposta Corretta)
\end{itemize}

\textbf{Esercizio 30}
Quanti sono gli anagrammi della parola PROBABILITÀ (fare come se l'accento non ci fosse)?
\begin{itemize}
    \item $39916800$
    \item $75600$
    \item $4989600$ (Risposta Corretta)
    \item $3628800$
    \item Nessuna delle altre risposte
\end{itemize}

\textbf{Esercizio 31}
Estraggo una carta da un mazzo di carte da Poker (52 carte). Considero gli eventi $A=$"estraggo una regina" e $B=$"estraggo una carta di cuori". Quanto vale $P(A|B)$?
\begin{itemize}
    \item $\frac{1}{13}$ (Risposta Corretta)
    \item $\frac{1}{4}$
    \item $\frac{1}{52}$
    \item $\frac{17}{52}$
\end{itemize}

\textbf{Esercizio 32}
Due dadi a sei facce sono così costruiti: il dado A ha due facce rosse e quattro facce blu; mentre il dado B ha tre facce rosse e tre facce blu. Se lancio i due dadi, qual è la probabilità che le facce siano dello stesso colore?
\begin{itemize}
    \item $\frac{1}{2}$ (Risposta Corretta)
    \item $\frac{1}{3}$
    \item Nessuna delle altre risposte
    \item $\frac{2}{3}$
    \item $\frac{1}{6}$
\end{itemize}

\textbf{Esercizio 33}
Siano $A, B$ eventi tali che $P(A|B)=\frac{1}{3}$. Quanto vale la probabilità condizionata $P(A^c|B)$?
\begin{itemize}
    \item Non la posso calcolare
    \item $\frac{1}{6}$
    \item $\frac{5}{6}$
    \item $\frac{1}{3}$
    \item $\frac{2}{3}$ (Risposta Corretta)
\end{itemize}

\textbf{Esercizio 34}
Mia nonna non era una gran cuoca, ma una volta al mese ci invitava tutti a pranzo da lei. Il suo repertorio di primi prevedeva $5$ formati di pasta (spaghetti, penne, farfalle, fettuccine, fusilli) e $3$ condimenti (pomodoro, ragù, carbonara), che abbinava a caso. Qual era la probabilità di non mangiare fusilli al ragù?
\begin{itemize}
    \item $\frac{2}{3}$
    \item Nessuna delle altre risposte
    \item $\frac{14}{15}$ (Risposta Corretta)
    \item $\frac{1}{15}$
    \item $\frac{4}{5}$
\end{itemize}

\textbf{Esercizio 35}
Durante un sondaggio, ad un gruppo di ragazzi è stato chiesto quale superpotere avrebbero voluto avere. Le risposte sono sintetizzate nella seguente tabella
\begin{center}
    \begin{tabular}{c|cc|c}
         & MASCHI & FEMMINE & Totale \\ \hline
        SAPER VOLAR & $30$ & $10$ & $40$ \\
        INVISIBILITÀ & $12$ & $32$ & $44$ \\
        ALTRO & $10$ & $6$ & $16$ \\ \hline
        Totale & $52$ & $48$ & $100$ \\
    \end{tabular}
\end{center}
Si scelga a caso un ragazzo di questo gruppo e si considerino gli eventi $A$="il ragazzo scelto è femmina" e $B$="il ragazzo scelto vorrebbe avere il dono dell'invisibilità". Allora $P(B|A)$ vale
\begin{itemize}
    \item $\frac{8}{11}$
    \item $\frac{5}{24}$
    \item $\frac{8}{25}$
    \item $\frac{2}{3}$ (Risposta Corretta)
    \item Nessuna delle altre risposte
\end{itemize}

\textbf{Esercizio 36}
Siano $A, B$ eventi \textit{indipendenti} tali che $P(A|B)=0.2$ e $P(B|A)=0.5$. Quanto vale la probabilità dell'evento $A\cup B$?
\begin{itemize}
    \item Non la posso calcolare 
    \item $0.6$ (Risposta Corretta)
    \item $o.4$
    \item $0.7$
    \item $0.8$
\end{itemize}

\textbf{Esercizio 37}
Un computer genera numeri casuali di $4$ cifre da $0000$ a $9999$ (estremi inclusi). Qual è la probabilità che il computer produca un numero che inizia e finisce con la cifra $1$?
\begin{itemize}
    \item $\frac{1}{100}$ (Risposta Corretta)
    \item $\frac{1}{10}$
    \item $\frac{1}{1000}$
    \item Nessuna delle altre risposte
    \item $\frac{1}{50}$
\end{itemize}

\textbf{Esercizio 38}
Nella mia scatola dei ricordi ho trovato un sacchetto di biglie di vetro Contiene ancora $21$ biglie: $8$ rosse, $7$ blu e $6$ verdi. Ne pesco una a caso. Qual è la probabilità che non sia né verde, né rossa?
\begin{itemize}
    \item $\frac{8}{21}$
    \item Nessuna delle altre risposte
    \item $\frac{1}{3}$ (Risposta Corretta)
    \item $\frac{5}{7}$
    \item $\frac{6}{21}$
\end{itemize}

\textbf{Esercizio 39}
Un'urna contiene $10$ palline rosse e $10$ palline viola. Si estraggono due palline \textit{senza reinserimento}. Qual è la probabilità che abbiano un colore diverso?
\begin{itemize}
    \item $\frac{10}{19}$ (Risposta Corretta)
    \item $\frac{1}{10}$
    \item $\frac{1}{2}$
    \item Nessuna delle altre risposte
    \item $\frac{2}{3}$
\end{itemize}

\textbf{Esercizio 40}
Siano $A, B$ eventi \textit{indipendenti} tali che $P(A)=\frac{1}{3}$ e $P(A\cap B^c)=\frac{1}{4}$. Qual è la probabilità dell'evento $B$?
\begin{itemize}
    \item $\frac{3}{4}$
    \item $\frac{2}{3}$
    \item Non la posso calcolare
    \item $\frac{1}{4}$ (Risposta esatta)
    \item $\frac{1}{2}$
\end{itemize}

\textbf{Esercizio 41}
Pesco una carta da un mazzo di carte da Poker (52 carte). Sapendo che è uscita una carta nera, qual è la probabilità che sia una figura?
\begin{itemize}
    \item $\frac{4}{13}$
    \item $\frac{6}{13}$
    \item $\frac{1}{2}$
    \item $\frac{3}{13}$ (Risposta Corretta)
    \item Nessuna delle altre risposte
\end{itemize}

\newpage
\subsection*{Secondo questionario}

\textbf{Argomenti}\\
Variabili aleatorie discrete, valore medio e varianza, variabili aleatorie notevoli, covarianza , vettori aleatori discreti\\

\textbf{Esercizio 42}
LeBron James è un famoso giocatore di basket. Negli USA, durante una partita di basket, un giocatore può commettere al massimo $6$ falli. Definiamo la variabile aleatoria $F$ come il numero di falli commessi da LeBron James durante una partita. La distribuzione di F è la seguente:\\
$p_F(0) = 0.15$, $p_F(1) = 0.27$, $p_F(2) = 0.35$, $p_F(3) = 0.15$, $p_F(4) = 0.04$, $p_F(5) = 0.01$, $p_F(6) = ?$\\
Quanto vale $P(F=6)$?

\begin{itemize}
    \item $0$
    \item $0.7$
    \item $Nessuna delle altre risposte$
    \item $0.0004$
    \item $0.03$ (Risposta Corretta)
\end{itemize}

\textbf{Esercizio 43}
Se si analizzano i dati numerici registrati nella maggior parte dei database reali, ci si accorge che la prima cifra di questi dati non è con uguale probabilità una delle cifre tra 1 e 9. La prima cifra più comune è 1, seguita da 2 e così via, in ordine, fino a 9 che è la cifra meno frequente. Questo fenomeno è noto come \textit{legge di Benford}.\\
Sia $D$ la variabile aleatoria che mi dà il valore della prima cifra dato un numerico preso a caso da un database. La distribuzione di $D$ è la seguente: \\
$p_D(1)=0.301$, $p_D(2)=0.176$, $p_D(3)=0.125$, $p_D(4)=0.097$, $p_D(5)=0.079$, $p_D(6)=0.067$, $p_D(7)=0.058$, $p_D(8)=0.051$, $p_D(9)=0.046$\\
Quanto vale la probabilità che $D$ sia dispari?

\begin{itemize}
    \item $0.609$ (Risposta Corretta)
    \item $0.391$
    \item Nessuna delle altre risposte
    \item $0.308$
    \item $0.563$
\end{itemize}

\textbf{Esercizio 44}
Una famosa hamburgeria di Boston offre hamburger con uno, due o tre medaglioni di carne. Sia $X$ il numero di medaglioni che un cliente qualunque ordina per il suo panino. La distribuzione di $X$ è la seguente: \\
$p_X(1)=0.4$, $p_X(2)=0.5$, $p_X(3)=0.1$\\
Si calcolano: $E(X)=1.7$ e $Var(X)=0.41$. Il costo totale di un hamburger è pari a 2 dollari per ogni medaglione di carne più un costo fisso di $1$ dollaro. Sia $T$ il prezzo dell'hamburger di un cliente qualunque. Quanto valgono $E(T)$ e $Var(T)$?

\begin{itemize}
    \item $4.4$ e $1.64$ (Risposta Corretta)
    \item Nessuna delle altre risposte
    \item $6.8$ e $1.82$
    \item $7.8$ e $0.82$
    \item $3.4$ e $2.64$
\end{itemize}

\textbf{Esercizio 45}
Sia $X\sim Bin(1, \frac{1}{3})$ e poniamo $Y=1-X$. Qual è la distribuzione di $Y$?\\

\begin{itemize}
    \item $Bin(1, \frac{1}{2})$
    \item $Bin(1, \frac{2}{3})$ (Risposta Corretta)
    \item Nessuna delle altre risposte
    \item $Bin(1, \frac{1}{3})$
    \item $Po(\frac{1}{3})$
\end{itemize}

\textbf{Esercizio 46}
Nel mio barattolo del caffè ci sono 10 capsule di decaffeinato, 5 di lungo e 15 di espresso. Pesco capsule in successione finché non trovo una capsula di caffè lungo. La variabile aleatoria $X$ è il numero di capsule pescate. La variabile aleatorie $X$ è geometrica?\\

\begin{itemize}
    \item No, perché l'esito di ogni estrazione non si può classificare come successo/insuccesso
    \item No, perché le estrazioni non sono tra loro indipendenti (Risposta Corretta)
    \item No, perché non è specificato un numero fissato di estrazioni
    \item Nessuna delle altre risposte
    \item Si, la variabile aleatoria $X$ è geometrica
\end{itemize}

\textbf{Esercizio 47}
Supponi di lanciare un dado 100 volte e di vincere \EUR{3} ogni volta che esce il punteggio 6. Quanti euro ti aspetti di vincere?

\begin{itemize}
    \item Nessuna delle altre risposte
    \item 300
    \item 50 (Risposta Corretta)
    \item 30
    \item 100
\end{itemize}

\textbf{Esercizio 48}
Bacco torna a casa dopo la festa dell'ultimo anno. Ha bevuto un po' troppo e non riesce a riconoscere la chiave di casa... Ha tre chiavi e prova e riprova una di queste tre, completamente a caso e senza scartare la chiave se non entra nella serratura, finché non riesce ad entrare. Qual è la probabilità che Bacco riesco ad aprire la porta entro il quarto tentativo (compreso)?

\begin{itemize}
    \item 0.33
    \item 0.80 (Risposta Corretta)
    \item Nessuna delle altre risposte
    \item 0.10
    \item 0.67
\end{itemize}

\textbf{Esercizio 49}
Un algoritmo estrapola una certa informazione da un dataset. L'algoritmo sbaglia nell'estrapolare questa informazione con probabilità $\frac{1}{200}$. Si testa l'algoritmo su 1000 dataset. Utilizzando l'approssimazione di Poisson, calcolare la probabilità che l'algoritmo sia in errore almeno una volta. Quanto vale questa probabilità?

\begin{itemize}
    \item 0.5643
    \item 0.0337 
    \item 0.0067
    \item 0.9933 (Risposta Corretta)
    \item Nessuna delle altre risposte
\end{itemize}

\textbf{Esercizio 50}
Siano $X$ e $Y$ due variabili aleatorie discrete con densità congiunta $p_{XY}$ illustrata dalla tabella\\
\begin{center}
    \renewcommand{\arraystretch}{1.75}
    \begin{tabular}{cc|ccc||c}
    \multicolumn{2}{c}{\multirow{2}{*}{$p_{XY}$}} & \multicolumn{3}{c}{Y} & \\
     & -1 & 0 & 1 & $p_X$ \\ \hline  
    \multirow{2}{*}{X} & -1 & 0 & $\frac{1}{6}$ & $\frac{1}{3}$ & $\frac{1}{2}$ \\
     & 1 & $\frac{1}{6}$ & $\frac{1}{6}$ & $\frac{1}{6}$ & $\frac{1}{2}$ \\ \hline \hline
     & $p_Y$ & $\frac{1}{6}$ & $\frac{1}{3}$ & $\frac{1}{2}$ & 1 \\
\end{tabular}
\end{center}
Quanto vale $P(XY\geq 0 | X < 0)$ ?

\begin{itemize}
    \item $\frac{1}{3}$ (Risposta Corretta)
    \item Non la posso calcolare 
    \item 0
    \item $\frac{1}{6}$
    \item $\frac{1}{2}$
\end{itemize}

\textbf{Esercizio 51}
Siano $X$ e $Y$ due variabili aleatorie discrete con densità congiunta $p_{XY}$ illustrata dalla tabella\\
\begin{center}
    \renewcommand{\arraystretch}{1.75}
    \begin{tabular}{cc|ccc||c}
    \multicolumn{2}{c}{\multirow{2}{*}{$p_{XY}$}} & \multicolumn{3}{c}{Y} & \\
     & -1 & 0 & 1 & $p_X$ \\ \hline  
    \multirow{2}{*}{X} & -1 & 0 & $\frac{1}{6}$ & $\frac{1}{3}$ & $\frac{1}{2}$ \\
     & 1 & $\frac{1}{6}$ & $\frac{1}{6}$ & $\frac{1}{6}$ & $\frac{1}{2}$ \\ \hline \hline
     & $p_Y$ & $\frac{1}{6}$ & $\frac{1}{3}$ & $\frac{1}{2}$ & 1 \\
\end{tabular}
\end{center}
Quanto vale il valore medio di $|X+Y|$ ?

\begin{itemize}
    \item $\frac{2}{3}$ (Risposta Corretta)
    \item $\frac{5}{6}$ 
    \item $\frac{1}{3}$
    \item Nessuna delle altre risposte
    \item $\frac{1}{6}$
\end{itemize}

\textbf{Esercizio 52}
Consideriamo le varaibili aleatorie X e Y tali che $Var(X)=3$, $Var(Y)=5$ e $Cov(X,Y)=-1$. Quanto vale $Var(X+Y)$?

\begin{itemize}
    \item Nessuna delle altre risposte
    \item 10
    \item 7
    \item 9
    \item 6 (Risposta Corretta)
\end{itemize}

\textbf{Esercizio 53}
Sia X una variabile aleatoria discreta tale che $P(X=0)=\frac{1}{4}$, $P(X=1)=\frac{1}{4}$ e $P(X=2)=\frac{1}{2}$. Allora la varianza di X vale

\begin{itemize}
    \item $\frac{9}{16}$ (Risposta Corretta)
    \item $\frac{11}{16}$
    \item $\frac{9}{4}$
    \item Nessuna delle altre risposte
    \item 1
\end{itemize}

\textbf{Esercizio 54}
Archimede Pitagorico ha sviluppato una app per smartphone. Ha notato che il $3\%$ degli utenti che scaricano al sua app la aggiornano quello stesso giorno. Sia $N$ il primo degli utenti che, la scorsa settimana, ha scaricato e aggiornato il giorno stesso la app di Archimede. Quanto vale $P(N>10)$?

\begin{itemize}
    \item Nessuna delle altre risposte
    \item 0.3
    \item 0.023
    \item 0.737 (Risposta Corretta)
    \item 0.263
\end{itemize}

\textbf{Esercizio 55}
Consideriamo le variabili aleatorie X e Y tali che $Var(X)=5$ e $Var(Y)=3$. Quanto vale $Var(X+Y)$?

\begin{itemize}
    \item Non ci sono sufficienti informazioni per poter rispondere (Risposta Corretta)
    \item 12
    \item 10
    \item 8
    \item 14
\end{itemize}

\textbf{Esercizio 56}
Sia X una variabile aleatoria discreta tale che $P(X=0)=P(X=-1)=\frac{1}{4}$ e $P(X=2)=\frac{1}{2}$. Definiamo $Z=2X^3-X^2$. Allora $P(Z=3)$ vale

\begin{itemize}
    \item Nessuna delle altre risposte
    \item $\frac{1}{4}$
    \item 0 (Risposta Corretta)
    \item $\frac{3}{4}$
    \item $\frac{1}{2}$
\end{itemize}

\textbf{Esercizio 57}
La variabile aleatoria X ha distribuzione binomiale con media 1 e varianza $\frac{2}{3}$. Quanto vale $P(X=2)$?

\begin{itemize}
    \item $\frac{2}{9}$ (Risposta Corretta)
    \item $\frac{2}{3}$
    \item $\frac{4}{9}$
    \item $\frac{1}{3}$
    \item Non ci sono informazioni sufficienti per poter rispondere
\end{itemize}

\textbf{Esercizio 58}
Siano X e Y due variabili aleatorie discrete con densità congiunta $p_{XY}$ illustrata in tabella
\begin{center}
    \renewcommand{\arraystretch}{1.25}
    \begin{tabular}{cc|cccc}
    \multicolumn{2}{c}{\multirow{2}{*}{$p_{XY}$}} & \multicolumn{4}{c}{Y} \\
     & & $0$ & $1$ & $2$ & $3$ \\ \hline  
    \multirow{3}{*}{X} & $-1$ & $0.05$ & $0.15$ & $0.05$ & $0.10$ \\
     & $0$ & $0.10$ & $0.05$ & $0.05$ & $0.05$ \\
     & $1$ & $0.10$ & $0.20$ & $0.05$ & $0.05$ \\
    \end{tabular}
\end{center}
Quanto vale $P(X<0)$?

\begin{itemize}
    \item Nessuna delle risposte precedenti
    \item $0.05$
    \item $0.60$
    \item $0.35$ (Risposta Corretta)
    \item $0.25$
\end{itemize}

\textbf{Esercizio 59}
Siano X e Y due variabili aleatorie discrete con densità congiunta $p_{XY}$ illustrata in tabella
\begin{center}
    \renewcommand{\arraystretch}{1.75}
    \begin{tabular}{cc|ccc}
    \multicolumn{2}{c}{\multirow{2}{*}{$p_{XY}$}} & \multicolumn{3}{c}{Y} \\
     & & $0$ & $1$ & $2$ \\ \hline  
    \multirow{2}{*}{X} & $-1$ & $\frac{1}{6}$ & $\frac{1}{6}$ & $\frac{1}{6}$\\
     & $1$ & $0$ & $\frac{1}{2}$ & $0$\\
    \end{tabular}
\end{center}
Quanto vale il valore medio di $X^2Y$?

\begin{itemize}
    \item 1 (Risposta Corretta)
    \item $-\frac{1}{3}$
    \item $\frac{4}{3}$
    \item 0
    \item Nessuna delle altre risposte
\end{itemize}

\textbf{Esercizio 60}
Sia X una variabile aleatoria che assume i valori 1, 2 e 3, tutti con la stessa probabilità $\frac{1}{3}$. Allora $E(X^3)$ vale

\begin{itemize}
    \item $\frac{2}{3}$
    \item $12$ (Risposta Corretta)
    \item $8$
    \item Nessuna delle risposte precedenti
    \item $\frac{14}{3}$
\end{itemize}

\textbf{Esercizio 61}
Nella cittadina di Delft il numero giornaliero di incidenti in bicicletta ha distribuzione Poisson di parametro 3. Qual è la probabilità che in un dato giorno ci siano esattamente due incidenti? 

\begin{itemize}
    \item $0.1992$
    \item Nessuna delle altre risposte
    \item $0.2240$ (Risposta Corretta)
    \item $0.0498$
    \item $0.1494$
\end{itemize}

\textbf{Esercizio 62}
Mr. Muscolo si allena provando dei tiri liberi. Ad ogni lancio la probabilità di fare canestro è $0.75$. Assumendo che i risultati dei tiri liberi siano indipendenti tra loro, dire quale delle seguenti è una variabile aleatoria binomiale:

\begin{itemize}
    \item Mr. Muscolo lancia ripetutamente tiri liberi. La variabile aleatoria X è il numero di tiri che vanno a canestro
    \item Mr. Muscolo lancia finché non fa canestro. La variabile X è il numero di tiri effettuati 
    \item Mr. Muscolo lancia 5 tiri liberi. La variabile aleatoria X è il numero di tiri che vanno a canestro  (Risposta Corretta)
    \item Mr. Musscolo lancia tiri liberi finché non fa 5 canestri. La variabile aleatoria X è il numero di tiri effettuati
    \item Nessuna delle altre risposte
\end{itemize}

\textbf{Esercizio 63}
Si consideri la variabile aleatoria $X\sim Po(\lambda)$. Si sa che $P(X=0)=\frac{1}{5}$. Allora il parametro $\lambda$ vale

\begin{itemize}
    \item $ln(\frac{5}{4}$
    \item $\frac{1}{5}$
    \item $ln 5$ (Risposta Corretta)
    \item Non lo posso calcolare
    \item $\frac{4}{5}$
\end{itemize}

\textbf{Esercizio 64}
In un gioco di fortuna un turno consiste nel lanciare un dado 12 volte. Sia X il numero di volte che si è ottenuto il punteggio 1 durante un turno. Quanto valgono (circa) media e varianza di X?

\begin{itemize}
    \item $0.17$ e $0.14$
    \item Nessuna delle altre risposte
    \item $2$ e $0.14$
    \item $2$ e $1.67$ (Risposta Corretta)
    \item $0.17$ e $1.67$
\end{itemize}

\textbf{Esercizio 65}
Nel gioco del Lotto, per ogni ruota, vengono estratti 5 numeri tra 1 e 90 senza reimmissione. La probabilità che un certo numero esca su una determinata ruota è $\frac{1}{18}$.\\
La mia anziana vicina gioca al Lotto: ha sognato il marito defunto, che le ha suggerito di giocare il numero 56 sulla ruota di Milano. Qual è circa la probabilità che il 56 esca per la prima volta entro la terza estrazione a partire da oggi?
\begin{itemize}
    \item $0.2044$
    \item $0.1578$ (Risposta Corretta)
    \item $0.8424$
    \item $0.0496$
    \item Nessuna delle risposte precedenti
\end{itemize}

\textbf{Esercizio 66}
Siano X e Y due variabili aleatorie discrete con densità congiunta $p_{XY}$ illustrata in tabella
\begin{center}
    \renewcommand{\arraystretch}{1.25}
    \begin{tabular}{cc|cccc}
    \multicolumn{2}{c}{\multirow{2}{*}{$p_{XY}$}} & \multicolumn{4}{c}{Y} \\
     & & $0$ & $1$ & $2$ & $3$ \\ \hline  
    \multirow{3}{*}{X} & $-1$ & $0.05$ & $0.15$ & $0.05$ & $0.10$ \\
     & $0$ & $0.10$ & $0.05$ & $0.05$ & $0.05$ \\
     & $1$ & $0.10$ & $0.20$ & $0.05$ & $0.05$ \\
    \end{tabular}
\end{center}
Quanto vale $P(X=1|XY=0)$?

\begin{itemize}
    \item $0.25$ (Risposta Corretta)
    \item $0.10$
    \item Nessuna delle risposte precedenti
    \item $0.15$
    \item $0.40$
\end{itemize}

\textbf{Esercizio 67}
Siano X e Y due variabili aleatorie discrete con densità congiunta $p_{XY}$ illustrata in tabella
\begin{center}
    \renewcommand{\arraystretch}{1.75}
    \begin{tabular}{cc|ccc}
    \multicolumn{2}{c}{\multirow{2}{*}{$p_{XY}$}} & \multicolumn{3}{c}{Y} \\
     & & $0$ & $1$ & $2$ \\ \hline  
    \multirow{2}{*}{X} & $-1$ & $\frac{1}{6}$ & $\frac{1}{6}$ & $\frac{1}{6}$\\
     & $1$ & $0$ & $\frac{1}{2}$ & $0$\\
    \end{tabular}
\end{center}
Quanto vale il valore medio di $|X|Y$?

\begin{itemize}
    \item 0
    \item Nessuna delle altre risposte
    \item $-\frac{1}{3}$
    \item $\frac{4}{3}$
    \item 1 (Risposta Corretta)
\end{itemize}

\textbf{Esercizio 68}
Si consideri la variabile aleatoria $X\sim Bin(n, p)$. Allora $P(0<X\geq 1)$ vale

\begin{itemize}
    \item $np(1-p)^{n-1}$ (Risposta Corretta)
    \item $(1-p)^n+np(1-p)^{n-1}$ 
    \item $(1-p)^n$
    \item Nessuna delle risposte precedenti
    \item $p(1-p)^{n-1}$
\end{itemize}

\textbf{Esercizio 69}
Papaerina è molto socievole e passa molto tempo al telefono con amici e familiari. Effettua quotidianamente un numero di chiamate descritto da una variabile aleatoria di Poisson di parametro $4.6$. Qual è la probabilità che oggi Paperina effettui almeno una chiamata?

\begin{itemize}
    \item Nessuna delle altre risposte
    \item $0.01$ 
    \item $0.046$
    \item $0.99$ (Risposta Corretta)
    \item $0.22$
\end{itemize}

\textbf{Esercizio 70}
Sia X una variabile aleatoria discreta tale che $P(X=0)=P(X=-1)=\frac{1}{4}$ e $P(X=2)=\frac{1}{2}$. Definiamo $Y=X-X^2$. Allora $P(Y=0)$ vale

\begin{itemize}
    \item $\frac{1}{2}$
    \item $\frac{3}{4}$
    \item Nessuna delle altre risposte
    \item 0
    \item $\frac{1}{4}$ (Risposta Corretta)
\end{itemize}

\textbf{Esercizio 71}
Mr. Muscolo si allena provando dei tiri liberi. Ad ogni lancio la probabilità di fare canestro è $0.75$. Assumendo che i risultati dei tiri liberi siano indipendenti tra loro, dire quale delle seguenti è una variabile aleatoria geometrica:

\begin{itemize}
    \item Mr. Muscolo lancia ripetutamente tiri liberi. La variabile aleatoria X è il numero di tiri che vanno a canestro
    \item Mr. Muscolo lancia finché non fa canestro. La variabile X è il numero di tiri effettuati (Risposta Corretta)
    \item Mr. Muscolo lancia 5 tiri liberi. La variabile aleatoria X è il numero di tiri che vanno a canestro
    \item Mr. Musscolo lancia tiri liberi finché non fa 5 canestri. La variabile aleatoria X è il numero di tiri effettuati
    \item Nessuna delle altre risposte
\end{itemize}

\textbf{Esercizio 72}
Siano X e Y due variabili aleatorie discrete con densità congiunta $p_{XY}$ illustrata in tabella
\begin{center}
    \renewcommand{\arraystretch}{1.75}
    \begin{tabular}{cc|ccc}
    \multicolumn{2}{c}{\multirow{2}{*}{$p_{XY}$}} & \multicolumn{3}{c}{Y} \\
     & & $0$ & $1$ & $2$ \\ \hline  
    \multirow{2}{*}{X} & $-1$ & $\frac{1}{6}$ & $\frac{1}{6}$ & $\frac{1}{6}$\\
     & $1$ & $\frac{1}{4}$ & $0$ & $\frac{1}{4}$\\
    \end{tabular}
\end{center}
Quanto vale $P(X>0|Y$ pari$)$?

\begin{itemize}
    \item $\frac{1}{2}$
    \item Nessuna delle altre risposte
    \item $-\frac{1}{3}$
    \item $\frac{2}{5}$
    \item $\frac{3}{5}$ (Risposta Corretta)
\end{itemize}

\textbf{Esercizio 73}
Siano X e Y due variabili aleatorie discrete con densità congiunta $p_{XY}$ illustrata in tabella
\begin{center}
    \renewcommand{\arraystretch}{1.75}
    \begin{tabular}{cc|ccc}
    \multicolumn{2}{c}{\multirow{2}{*}{$p_{XY}$}} & \multicolumn{3}{c}{Y} \\
     & & $0$ & $1$ & $2$ \\ \hline  
    \multirow{2}{*}{X} & $-1$ & $\frac{1}{6}$ & $\frac{1}{6}$ & $\frac{1}{6}$\\
     & $1$ & $0$ & $\frac{1}{2}$ & $0$\\
    \end{tabular}
\end{center}
Quanto vale il valore medio di $XY$?

\begin{itemize}
    \item 0 (Risposta Corretta)
    \item Nessuna delle altre risposte
    \item $-\frac{1}{3}$
    \item $\frac{4}{3}$
    \item 1 
\end{itemize}

\textbf{Esercizio 74}
Si consideri la variabile aleatoria $X\sim Ge(p)$. Allora $P(X\geq 2$ vale

\begin{itemize}
    \item $1-(1-p)^2$
    \item Nessuna delle altre risposte
    \item $p$
    \item $(1-p)^2$
    \item $1-p$ (Risposta Corretta)
\end{itemize}

\textbf{Esercizio 75}
Sia X una variabile aleatoria discreta tale che $P(X=0)=P(X=-1)=\frac{1}{4}$ e $P(X=2)=\frac{1}{2}$. Definiamo $Y=X-X^2$. Allora l'alfabeto di Y è

\begin{itemize}
    \item $\{0, -2, -6\}$
    \item $\{0, -2, 2\}$
    \item Nessuna delle altre risposte
    \item $\{0, -6\}$
    \item $\{0, -2\}$ (Risposta Corretta)
\end{itemize}

\textbf{Esercizio 76}
Si lanciano contemporaneamente tre monete. La variabile aleatoria X rappresenta il numero di monete che hanno dato testa come risultato del lancio. La distribuzione di X è la seguente :\\
$p_X(0)=\frac{1}{8}$, $p_X(1)=\frac{3}{8}$, $p_X(2)=\frac{3}{8}$, $p_X(3)=\frac{1}{8}$\\
Quanto vale la varianza di X?

\begin{itemize}
    \item $\frac{9}{4}$
    \item $\frac{3}{4}$ (Risposta Corretta)
    \item Nessuna delle altre risposte
    \item $\frac{3}{2}$
    \item $3$
\end{itemize}

\textbf{Esercizio 77}
Archimede Pitagorico ha sviluppato una app per smartphone. Ha notato che il $3\%$ degli utenti che scaricano al sua app la aggiornano quello stesso giorno. La scorsa settimana ci sono stati 5000 download. Sia $N$ il numero di utenti che, la scorsa settimana, ha scaricato e aggiornato il giorno stesso la app di Archimede. Che tipo di variabile aleatoria è N ?

\begin{itemize}
    \item Variabile aleatoria binomiale (Risposta Corretta)
    \item Nessuna delle altre risposte
    \item Variabile aleatoria geometrica
    \item Variabile aleatoria di Bernoulli
    \item Variabile aleatoria di Poisson
\end{itemize}

\textbf{Esercizio 78}
Si consideri la variabile aleatoria $X\sim Bin(1000, \frac{1}{500})$. Usando l'\textit{approssimazione di Poisson}, calcolare $P(X>0)$. Quanto vale questa probabilità?

\begin{itemize}
    \item $0.2707$
    \item $0.8647$ (Risposta Corretta)
    \item $0.1353$
    \item $1$
    \item Nessuna delle altre risposte
\end{itemize}

\textbf{Esercizio 79}
La variabile aleatoria X ha distribuzione binomiale con media 1 e varianza $\frac{4}{5}$. Quanto vale $P(X=2)$?

\begin{itemize}
    \item $0.25$
    \item $0.10$
    \item $0.20$ (Risposta Corretta)
    \item $0.30$
    \item Non ci sono informazioni sufficienti per poter rispondere
\end{itemize}

\textbf{Esercizio 80}
Due biologi incrociano due piante. In ciascuna il gene che determina il colore è della forma \textbf{Gg}. Questo significa che ogni pianta nata dall'incrocio ha probabilità $0.25$ di ereditare il gene \textbf{gg}, che rende la pianta albina. Si assuma che la trasmissione dei geni sia indipendente per ogni pianta generata. Si controllano una dopo l'altra le piante nate da questo incrocio. Qual è la probabilità che la prima pianta albina osservata sia la sesta pianta che viene controllata?

\begin{itemize}
    \item $0.0007$
    \item Nessuna delle altre risposte
    \item $0.0593$ (Risposta Corretta)
    \item $0.0044$
    \item $0.3560$
\end{itemize}

\textbf{Esercizio 81}
La variabile aleatoria X ha densità discreta della forma $P(X=k)=e^{-4}\frac{4^k}{k!}$, per ogni $k\in \mathbf{N}_0$. Quanto vale $E(X^2)$ ?

\begin{itemize}
    \item Nessuna delle altre risposte
    \item 20 (Risposta Corretta)
    \item 4
    \item 16
    \item 8
\end{itemize}

\textbf{Esercizio 82}
Siano X e Y due variabili aleatorie discrete con densità congiunta $p_{XY}$ illustrata in tabella
\begin{center}
    \renewcommand{\arraystretch}{1.75}
    \begin{tabular}{cc|ccc}
    \multicolumn{2}{c}{\multirow{2}{*}{$p_{XY}$}} & \multicolumn{3}{c}{Y} \\
     & & $0$ & $1$ & $2$ \\ \hline  
    \multirow{2}{*}{X} & $-1$ & $\frac{1}{6}$ & $\frac{1}{6}$ & $\frac{1}{6}$\\
     & $1$ & $0$ & $\frac{1}{2}$ & $0$\\
    \end{tabular}
\end{center}
Quanto vale il valore medio di $|X|+Y$?

\begin{itemize}
    \item Nessuna delle altre risposte
    \item $-1$
    \item $2$ (Risposta Corretta)
    \item $1$
    \item $0$
\end{itemize}

\textbf{Esercizio 83}
Il numero di particelle radioattive rilevate da un contatore Geiger è descritto da una variabile aleatoria di Poisson. In media nell'arco di 1 millisecondo, il contatore rileva 4 particelle. Qual è la probabilità che in 1 millisecondo il contatore non rilevi alcuna particella?

\begin{itemize}
    \item $0.0183$ (Risposta Corretta)
    \item $0.0025$
    \item Non ci sono informazioni sufficienti per poter rispondere
    \item $0.1606$
    \item $0.9817$
\end{itemize}

\textbf{Esercizio 84}
La distribuzione Bin(1000, 0.004) può essere approssimata con la distribuzione:

\begin{itemize}
    \item Nessuna delle altre risposte
    \item Ge(0.0004)
    \item Po(2.5)
    \item Be(0.0004)
    \item Po(4) (Risposta Corretta)
\end{itemize}

\newpage
\subsection*{Terzo questionario}

\textbf{Argomenti}\\
Variabili aleatorie assolutamente continue, valore medio, varianza e indipendenza,, variabili aleatorie continue notevoli, funzione di distribuzione, trasformazioni affini, legge dei grandi numeri e teorema del limite centrale, approssimazione normale\\

\textbf{Esercizio 85}
Sia $X$ una variabile aleatoria assolutamente continua con funzione di distribuzione data da
\[
F_X(x)=
\begin{cases}
    0 & x<0\\
    \frac{1}{24}(2x^2+x^4) & 0\leq x \leq 2 \\
    1 & x>2
\end{cases}
\]
Quanto vale $P(X\leq 1)$?
 
\begin{itemize}
    \item $\frac{1}{8}$ (Risposta Corretta)
    \item $\frac{1}{3}$
    \item $\frac{1}{4}$
    \item $\frac{1}{16}$
    \item Nessuna delle altre risposte
\end{itemize}

\textbf{Esercizio 86}
Sia $Y$ una variabile aleatoria assolutamente continua con funzione di distribuzione data da
\[
F_Y(y)=
\begin{cases}
    0 & y<0\\
    \sqrt{y} & 0\leq y \leq 1 \\
    1 & y>1
\end{cases}
\]
Quanto vale $P(\frac{1}{16}\leq Y \leq \frac{1}{4})$?
 
\begin{itemize}
    \item $\frac{1}{2}$
    \item $\frac{1}{4}$ (Risposta Corretta)
    \item Nessuna delle altre risposte
    \item $0$
    \item $\frac{7}{96}$
\end{itemize}

\textbf{Esercizio 87}
Siano $X\sim Po(2)$ e $Y\sim Po(6)$ due variabili aleatorie indipendenti. Quanto vale $Var(2 - 3X + Y)$?

\begin{itemize}
    \item $26$
    \item $14$
    \item Nessuna delle altre risposte
    \item $24$ (Risposta Corretta)
    \item $12$
\end{itemize}

\textbf{Esercizio 88}
Sia $X$ una variabile aleatoria esponenziale con valore medio $\frac{1}{5}$. Quanto vale $P(X>5|X>3)$?

\begin{itemize}
    \item $1-e^{-10}$
    \item $e^{-2}$
    \item $1-e^{-2}$
    \item Nessuna delle altre risposte
    \item $e^{-10}$ (Risposta Corretta)
\end{itemize}

\textbf{Esercizio 89}
Sia $X$ una variabile aleatoria assolutamente continua con densità di probabilità data da
\[
f_x(x)=
\begin{cases}
    \frac{1}{6}(x+x^3) & 0\leq x\leq 2\\
    0 & altrimenti
\end{cases}
\]
Quanto vale $P(X\leq 1)$?
 
\begin{itemize}
    \item $\frac{1}{3}$
    \item $\frac{1}{4}$
    \item $\frac{1}{8}$ (Risposta Corretta)
    \item $\frac{1}{16}$
    \item Nessuna delle altre risposte
\end{itemize}

\textbf{Esercizio 90}
Siano $X_1$, ..., $X_{100}$ variabili aleatorie i.i.d con distribuzione $U(0,6)$. Quale delle seguenti variabili ha una distribuzione che può essere approssimata con la distribuzione di una normale centrata standard?

\begin{itemize}
    \item $\frac{\overline{X}_{100}-3}{3}$
    \item $10\cdot \frac{\overline{X}_{100}-3}{\sqrt{3}}$ (Risposta Corretta)
    \item $\overline{X}_{100}$
    \item $\frac{\overline{X}_{100}-6}{\sqrt{3}}$
    \item Nessuna delle altre risposte
\end{itemize}

\textbf{Esercizio 91}
Si consideri $X\sim U(-\frac{1}{2}, \frac{1}{2}$. Il valore medio della variabile aleatoria $Y=2e^{2X}$ vale

\begin{itemize}
    \item $e - e^{-1}$ (Risposta Corretta)
    \item Nessuna delle altre risposte
    \item $e^{-1}$
    \item $0$
    \item $e$
\end{itemize}

\textbf{Esercizio 92}
Siano $X\sim N(6,5)$ e $Y\sim N(3,4)$ indipendenti. Quanto vale $P(X-Y>6)$?

\begin{itemize}
    \item $0.668$
    \item $0.0808$
    \item $0.244$
    \item $0.1587$ (Risposta Corretta)
    \item Nessuna delle altre risposte
\end{itemize}

\textbf{Esercizio 93}
Sia $X\sim Exp(2)$. Allora $E(X^2)$ vale

\begin{itemize}
    \item $2$
    \item Nessuna delle altre risposte
    \item $\frac{1}{2}$ (Risposta Corretta)
    \item $4$
    \item $\frac{1}{4}$
\end{itemize}

\textbf{Esercizio 93}
Sia $X$ una variabile aleatoria assolutamente continua con densità $f$. Siano inoltre $a,b$ due numeri reali tali che $a<b$. Allora

\begin{itemize}
    \item $P(X\in \{a,b\})=f(a)f(b)$
    \item $P(X\in \{a,b\})=0$ (Risposta Corretta)
    \item $P(X\in \{a,b\})=\int_{a}^{b}f(x) dx $
    \item $P(X\in \{a,b\})=f(a)+f(b)$
    \item Nessuna delle altre risposte
\end{itemize}

\textbf{Esercizio 94}
Siano $X_1$,...,$X_{81}$ variabili aleatorie i.i.d con distribuzione $Ge(\frac{1}{2})$. Quale delle seguenti variabili ha una distribuzione che può essere approssimata con la distribuzione di una normale centrata standard?

\begin{itemize}
    \item $9\cdot \frac{\overline{X}_{81}-1}{\sqrt{0.5}}$
    \item $9\cdot \frac{0.5\overline{X}_{81}-1}{\sqrt{0.5}}$ (Risposta Corretta)
    \item $\frac{0.5\overline{X}_{81}-1}{\sqrt{0.5}}$
    \item $\overline{X}_{81}$
    \item Nessuna delle altre risposte
\end{itemize}

\textbf{Esercizio 95}
Siano $X\sim N(-1,2)$,$Y\sim N(3,2)$ e $Z\sim N(0,1)$ indipendenti. Il valore di $x\in \mathbf{R}$ per cui si ottiene $P(X-Y\leq 0)=P(Z\leq x)$ è

\begin{itemize}
    \item $0$
    \item Nessuna delle altre risposte
    \item $2$ (Risposta Corretta)
    \item $-2$
    \item $4$
\end{itemize}

\textbf{Esercizio 96}
Si consideri $X\sim U(0,64)$. Il valore medio della variabile aleatoria $Y=\frac{3}{2}\sqrt{X}$ vale

\begin{itemize}
    \item $512$
    \item $32$
    \item $8$ (Risposta Corretta)
    \item $64$
    \item Nessuna delle altre risposte
\end{itemize}

\textbf{Esercizio 97}
Sia $X\sim Exp(1)$. Quale delle seguenti variabili è standardizzata (cioè ha media zero e varianza uno)?

\begin{itemize}
    \item Nessuna delle altre risposte
    \item $X-1$ (Risposta Corretta)
    \item $X+1$
    \item $\frac{X}{2}$
    \item $X$
\end{itemize}

\textbf{Esercizio 98}
Sia $X$ una variabile aleatoria esponenziale con valore medio $\frac{1}{6}$. Quanto vale $P(X>3|X<1)$?

\begin{itemize}
    \item Nessuna delle altre risposte (Risposta Corretta)
    \item $\frac{e^{-18}-e^{-6}}{e^{-18}}$
    \item $\frac{e^{-1}-e^{-3}}{1-e^{-3}}$
    \item $\frac{e^{-6}-e^{-18}}{1-e^{-18}}$
    \item $\frac{e^{-3}-e^{-1}}{e^{-3}}$
\end{itemize}

\textbf{Esercizio 99}
Sia $X$ una variabile aleatoria assolutamente continua con densità di probabilità data da
\[
f_x(x)=
\begin{cases}
    6(x+x^2) & 0\leq x\leq 1\\
    0 & altrimenti
\end{cases}
\]
Quanto vale $P(0\leq X\leq 0.5)$?
 
\begin{itemize}
    \item Nessuna delle altre risposte
    \item $0.5$ (Risposta Corretta)
    \item $0.6$
    \item $0.3$
    \item $0.4$
\end{itemize}

\textbf{Esercizio 100}
Sia $X$ una variabile aleatoria assolutamente continua con funzione di distribuzione data da
\[
F_X(x)=
\begin{cases}
    0 & x<-1\\
    \frac{1}{4}(x+1) & -1\leq x \leq 3 \\
    1 & x>3
\end{cases}
\]
Quanto vale $P(1\leq X \leq 2)$?
 
\begin{itemize}
    \item $\frac{1}{4}$ (Risposta Corretta)
    \item $\frac{3}{4}$
    \item $\frac{7}{8}$
    \item $\frac{1}{8}$
\end{itemize}

\textbf{Esercizio 101}
Sia $Y$ una variabile aleatoria assolutamente continua con funzione di distribuzione data da
\[
F_Y(y)=
\begin{cases}
    0 & y<0\\
    \sqrt{y} & 0\leq y \leq 1 \\
    1 & y>1
\end{cases}
\]
Quanto vale $P(Y \geq \frac{1}{25})$?

\begin{itemize}
    \item $\frac{1}{25}$
    \item $\frac{4}{5}$ (Risposta Corretta)
    \item $1$
    \item $\frac{1}{5}$
    \item Nessuna delle altre risposte
\end{itemize}

\textbf{Esercizio 102}
Siano $X\sim Bin(1,\frac{1}{4})$ e $Y\sim Bin(2,\frac{1}{4})$ indipendenti. Quale delle seguenti affermazione è falsa?

\begin{itemize}
    \item $E(X+Y)=\frac{3}{4}$
    \item Nessuan delle altre risposte (Risposta Corretta)
    \item $Var(X+Y)=\frac{9}{16}$
    \item $P(X=1)<P(Y=1)$
    \item $X+Y\sim Bin(3,\frac{1}{4}$
\end{itemize}

\textbf{Esercizio 103}
Sia $T$ una variabile aleatoria assolutamente continua con funzione di distribuzione data da
\[
F_T(t)=
\begin{cases}
    0 & t<0\\
    2t^2-t^4 & 0\leq t \leq 1 \\
    1 & t>1
\end{cases}
\]
Quanto vale $P(T \geq \frac{1}{2})$?

\begin{itemize}
    \item $\frac{7}{16}$
    \item $\frac{87}{160}$
    \item $\frac{9}{16}$ (Risposta Corretta)
    \item $\frac{1}{2}$
    \item Nessuna delle altre risposte
\end{itemize}

\textbf{Esercizio 104}
Consideriamo le variabili aleatorie $X$ e $Y$, indipendenti e tali che $Var(X)=1$ e $Var(Y)=4$. Quanto vale $Var(X+Y-4)$?

\begin{itemize}
    \item $5$ (Risposta Corretta)
    \item Non ci sono sufficienti informazioni per poter rispondere
    \item $4$
    \item $1$
    \item $3$
\end{itemize}

\textbf{Esercizio 105}
Sia $X\sim Exp(2)$. Quale delle seguenti variabili è standardizzata (cioè ha media zero e varianza uno)?

\begin{itemize}
    \item $2X-1$ (Risposta Corretta)
    \item $\frac{X-2}{4}$
    \item Nessunad delle altre risposte
    \item $\frac{X+2}{4}$
    \item $2X+1$
\end{itemize}

\textbf{Esercizio 106}
Sia $X_1, X_2, ...$ una successione di variabili aleatorie i.i.d con valore medio $1$ e varianza $\sigma^2$. Sia inoltre $Z\sim N(0,1)$. Preso $x\in \mathbf{R}$, quale delle seguenti relazioni, per $n$ sufficientemente grande, è una conseguenza del teorema del limite centrale?

\begin{itemize}
    \item $P(\overline{X}_n > \frac{1+\sigma x}{\sqrt{n}}) \approx P(Z\leq x)$
    \item Nessuna delle altre risposte
    \item $P(\overline{X}_n < \frac{1}{\sqrt{n}}) \approx \frac{1}{2}$
    \item $P(\overline{X}_n \leq \frac{\sigma x}{\sqrt{n}}+1) \approx P(Z\leq x)$ (Risposta Corretta)
    \item $P(\overline{X}_n > 1 + \sigma x) \approx P(Z\leq x)$
\end{itemize}

\textbf{Esercizio 107}
Sia $X$ una variabile aleatoria assolutamente continua con funzione di distribuzione data da
\[
F_X(x)=
\begin{cases}
    0 & x<0\\
    \frac{1}{2}(3x-x^3) & 0\leq x \leq 1 \\
    1 & x>1
\end{cases}
\]
Quanto vale $P(0 < X < \frac{1}{3})$?
 
\begin{itemize}
    \item $\frac{13}{27}$ (Risposta Corretta)
    \item $\frac{14}{27}$
    \item $\frac{1}{3}$
    \item Nessuna delle altre risposte
    \item $\frac{2}{3}$
\end{itemize}

\textbf{Esercizio 108}
Si consideri $X\sim U(-1,1)$. Il valore medio della variabile aleatoria $Y=4e^{2X}$ vale

\begin{itemize}
    \item $\frac{1}{e}$
    \item Nessuna delle altre risposte
    \item $0$
    \item $e-e^{-1}$
    \item $e^2-e^{-2}$ (Risposta Corretta)
\end{itemize}

\textbf{Esercizio 109}
Sia $X$ una variabile aleatoria assolutamente continua con densità di probabilità  data da
\[
f_X(x)=
\begin{cases}
    0.3 & 0\leq x\leq 1\\
    K & 1\leq x \leq 2 \\
\end{cases}
\]
Quanto vale la costante $K$?

\begin{itemize}
    \item $0.5$
    \item $0.7$ (Risposta Corretta)
    \item $0.2$
    \item $0.8$
    \item Nessuna delle altre risposte
\end{itemize}

\textbf{Esercizio 110}
Siano $X\sim Ge(\frac{1}{3})$ e $Y\sim Ge(\frac{1}{2})$ due variabili aleatorie indipendenti. Quanto vale $Var(4+X-3Y)$?

\begin{itemize}
    \item $16$
    \item $28$
    \item $24$ (Risposta Corretta)
    \item Nessuna delle altre risposte
    \item $12$
\end{itemize}

\textbf{Esercizio 111}
Si consideri $X\sim U(-1,1)$. Il valore medio della variabile aleatoria $Y=X^3$ vale

\begin{itemize}
    \item $\frac{1}{2}$
    \item $0$ (Risposta Corrette)
    \item Nessuna delle altre risposte
    \item $\frac{1}{8}$
    \item $\frac{1}{4}$
\end{itemize}

\textbf{Esercizio 112}
Sia $X\sim N(1,4)$ e sia $\Phi$ la funzione di distribuzione della gaussiana centrata standard. Allora, per $x\in \mathbf{R}$, $P(X\leq x)$ è uguale a

\begin{itemize}
    \item $\Phi(x-1)$
    \item $\Phi(\frac{x-1}{2})$ (Risposta Corretta)
    \item $\Phi(x)$
    \item $\Phi(\frac{x-1}{4})$
    \item Nessuna delle altre risposte
\end{itemize}

\textbf{Esercizio 113}
Sia $X$ una variabile aleatoria esponenziale con valore medio $\frac{1}{2}$. Quanto vale $P(X>6|X<4)$?

\begin{itemize}
    \item $1-e^{-2}$
    \item $e^{-4}$ (Risposta Corretta)
    \item $e^{-2}$
    \item Nessuna delle altre risposte
    \item $1-e^{-4}$
\end{itemize}

\textbf{Esercizio 114}
Sia $X$ una variabile aleatoria assolutamente continua con funzione di distribuzione data da
\[
F_x(x)=
\begin{cases}
    0 & x<0\\
    3x^2-2x^2 & 0\leq x\leq 1\\
    1 & x>1
\end{cases}
\]
Quanto vale $P(0.5\leq X\leq 1)$?
 
\begin{itemize}
    \item $0.5$ (Risposta Corretta)
    \item $1$
    \item $0.6$
    \item Nessuna delle altre risposte
    \item $0.4$
\end{itemize}

\textbf{Esercizio 115}
Siano $X_1$, ..., $X_{36}$ variabili aleatorie i.i.d con distribuzione $Bin(100,\frac{1}{5})$. Quale delle seguenti variabili ha una distribuzione che può essere approssimata con la distribuzione di una normale centrata standard?

\begin{itemize}
    \item $10\cdot \frac{\overline{X}_{36}-0.2}{\sqrt{0.16}}$
    \item $\overline{X}_{36}$
    \item $\frac{\overline{X}_{36}-20}{4}$
    \item $3\cdot \frac{\overline{X}_{36}-20}{2}$ (Risposta Corretta) 
    \item Nessuna delle altre risposte
\end{itemize}

\textbf{Esercizio 116}
Sia $X$ una variabile aleatoria assolutamente continua con densità di probabilità data da
\[
f_x(x)=
\begin{cases}
    \frac{1}{x} & 1\leq x\leq e\\
    0 & altrimenti
\end{cases}
\]
Quanto vale $P(\frac{e}{2}\leq X\leq e)$?
 
\begin{itemize}
    \item $ln2$ (Risposta Corretta)
    \item Nessuna delle altre risposte
    \item $\frac{e}{2}$
    \item $\frac{1}{2}$
    \item $e$
\end{itemize}

\end{document}