\documentclass{report}
\usepackage[utf8]{inputenc}
\usepackage{eurosym}

\author{Elia Pasquali}

\begin{document}

\begin{center}
    {\Large\textbf{Esercizi di Probabilità e Statistica}}
\end{center}
Questa raccolta di esercizi è tratta dai vari quiz online degli ultimi anni. 

\section*{Anno 19/20}
Gli esercizi di questa sezione sono stati ricontrollati e dovrebbero essere corretti. Le fonti di alcuni erano illeggibili e quindi la riscruttura non è completa.\\

\textbf{Esercizio 1}
Siano $A, B$ eventi \textit{indipendenti} tali che $P(A^c) = 0.1$ e $P(B) = 0.7$. Qual è la probabilità dell'evento $A \cap B^c$?
\begin{itemize}
    \item $0.63$
    \item $0.03$
    \item Non la posso calcolare
    \item $0.27$ (Risposta Corretta)
    \item $0.07$
\end{itemize}

\textbf{Esercizio 2}
Siano $A,B,C$ eventi \textit{indipendenti} tali che $P(A) = P(B) = P(C) = \frac{1}{2}$. Qual è la probabilità dell'evento $(A \cap B) \cap C$?
\begin{itemize}
    \item Non la posso calcolare
    \item $\frac{1}{8}$ (Risposta Corretta)
    \item $\frac{7}{8}$
    \item $\frac{3}{8}$ 
    \item $\frac{5}{8}$
\end{itemize}

\textbf{Esercizio 3}
Lanciamo un dado. Consideriamo i seguenti eventi: $E_\ge = $"esce un punteggio maggiore o uguale a 4", $E_i =$ "esce il punteggio $i$" (con $i = 1,..., 6)$ e $E =$ "esce un punteggio divisibile per 3". Quale delle seguenti affermazioni è \textit{falsa}?
\begin{itemize}
    \item Gli eventi $E_4$ ed $E$ sono incompatibili
    \item Gli eventi $E_1$, $E_2$, $E$ ed $E_\ge$ non sono una partizione
    \item Gli eventi $E_i$ con $i = 1,...,6$ sono una partizione
    \item Nessuna delle altre risposte  (Risposta corretta)
    \item Gli eventi $E_1$, $E_2$, $E_3$ ed $E_\ge$ non sono una partizione
\end{itemize}

\textbf{Esercizio 4}
Sia $A,B$ eventi tali che $P(A \cup B) = 0.4$. Qual è la probabilità dell'evento $A^c \cap B^c$?
\begin{itemize}
    \item Non la posso calcolare
    \item $0.3$
    \item $0.7$
    \item $0.4$
    \item $0.6$ (Risposta Corretta)
\end{itemize}

\textbf{Esercizio 5}
Holly si allena tirando ai rigori. Ad ogni tiro, indipendentemente dagli altri, ha la probabilità 0.8 di segnare. Qual è la probabilità che segni tutti i prossimi 3 tiri?
\begin{itemize}
    \item $0.512$ (Risposta Corretta)
    \item $0.008$
    \item $0.6$
    \item Nessuna delle altre risposte
    \item $1$
\end{itemize}

\textbf{Esercizio 6}
In una classe di 8 alunni, 3 non hanno fatto i compiti. Se l'insegnante ne sceglie 2 a caso, qual è la probabilità che entrambi abbiano svolto i compiti?
\begin{itemize}
    \item $\frac{10}{14}$
    \item $\frac{5}{14}$ (Risposta Corretta)
    \item Nessuna delle altre risposte
    \item $\frac{5}{28}$
    \item $\frac{5}{8}$
\end{itemize}

\textbf{Esercizio 7}
La mia amica Martina adora lo yoga e segue un corso avanzato tre volte alla settimana. Il 60\% dei partecipanti a questo corso sono donne e la probabilità che una donna abbia un abbigliamento sportivo sui toni del viola è del 10\%. Sapendo che si è scelta una donna della classe, qual è la probabilità che indossi qualcosa di viola?
\begin{itemize}
    \item $\frac{3}{5}$
    \item Nessuna delle altre risposte
    \item $\frac{1}{6}$
    \item $\frac{7}{10}$
    \item $\frac{1}{10}$ (Risposta Corretta)
\end{itemize}

\textbf{Esercizio 8}
La ruota della roulette ha 18 sezioni rosse, 18 nere e 1 verde. Se si punta \euro 1 sul colore rosso (risp. nero) e la pallina si ferma in una sezione rossa (risp. nera), allora si vince €1; altrimenti si perde la putnata. Sia $X$ il guadagno di un giocatore che fa questo tipo di scommessa. La distribuzione di $X$ è la seguente:\\
$px(-1) = \frac{19}{37}$ (perdita) , $px(1) = \frac{18}{37}$ (vincita).\\
Si calcolano: $E(X)\approx -0.03$ e $Var(X)\approx 1$. Supponiamo che un giocatore decida di puntare \euro 5, invece che \euro 1, cosicchè potrebbe vincere o perdere \euro 5. Sia $Z$ la vincita derivante da una puntata di \euro 5 sul nero. Quanto vale (Circa) il valore medio di $Z$?
\begin{itemize}
    \item $\frac{3}{5}$
    \item Nessuna delle altre risposte (Risposta Corretta)
    \item $\frac{1}{6}$
    \item $\frac{7}{10}$
    \item $\frac{1}{10}$
\end{itemize}

\textbf{Esercizio 9}
Estraggo una carta da un mazzo di carte da Poker (52 carte). Qual è la probabilità di sceglie picche sapendo che la carta scelta è nera?
\begin{itemize}
    \item $\frac{1}{26}$
    \item $\frac{1}{13}$
    \item $\frac{1}{4}$
    \item $\frac{1}{2}$ (Risposta Corretta)
    \item Nessuna delle altre risposte
\end{itemize}

\textbf{Esercizio 10}
Sia $X$ una variabile aleatoria discreta tale che $P(X=0)=P(X=1)=\frac{1}{4}$ e $P(X=-2)=\frac{1}{2}$. Definiamo $Z=2X^3-X^2$. Allora $P(Z=1)$ vale
\begin{itemize}
    \item $0$
    \item Nessuna delle altre risposte
    \item $\frac{1}{2}$
    \item $\frac{1}{4}$ (Risposta Corretta)
    \item $\frac{1}{2}$
\end{itemize}

\textbf{Esercizio 11}
Un barattolo di caramelle contiene 6 gelatine rosse, 4 gelatine verdi e 4 gelatine blu. Ne prendo due a caso, qual è la probabilità che siano entrambe blu?
\begin{itemize}
    \item $\frac{3}{91}$
    \item $\frac{8}{14}$
    \item Nessuna delle altre risposte
    \item $\frac{12}{91}$
    \item $\frac{6}{91}$ (Risposta Corretta)
\end{itemize}

\textbf{Esercizio 12}
Sia $X$ una variabile aleatoria discreta tale che $P(X=0)=\frac{1}{4}, P(X=1)=\frac{1}{4}$ e $P(X=2)=\frac{1}{2}$. Allora vale
\begin{itemize}
    \item $\frac{9}{4}$ (Risposta Corretta)
    \item $\frac{9}{16}$
    \item $1$
    \item $\frac{11}{16}$
    \item Nessuna delle altre risposte
\end{itemize}

\textbf{Esercizio 13}
Lancio due dadi. Qual è la probabilità che la somma dei due punteggia sia maggiore di 3?
\begin{itemize}
    \item Nessuna delle altre risposte
    \item $\frac{1}{12}$
    \item $\frac{35}{36}$
    \item $\frac{1}{18}$
    \item $\frac{11}{12}$ (Risposta Corretta)
\end{itemize}

\textbf{Esercizio 14}
Un'urna contiene 2 palline rosse, 3 palline gialle e 2 palline blu. Pesco due palline \textit{con reinserimento}. Qual è la probabilità che nessuna pallina sia blu?
\begin{itemize}
    \item $\frac{1}{2}$
    \item $\frac{10}{21}$
    \item $\frac{20}{21}$
    \item $\frac{25}{49}$ (Risposta Corretta)
    \item Nessuna delle altre risposte
\end{itemize}

\textbf{Esercizio 15} \underline{Nota}: alcune parti del testo erano indecifrabili\\
Se si analizzano dei dati ... ci si accorge che la prima cifra di questi dati non è con uguale probabilità una delle cifre tra 1 e 9. La prima cifra più comune è 1, seguita da 2 e così via, in ordine, fino a 9 che è la prima cifra meno frequente. Questo fenomeno è noto con il nome \textit{legge di Benford}. Sia $D$ la variabile aleatoria che mi da il valore di un dato numerico da questa legge, la ... di $D$ è la seguente:\\
$p_D(1)=0.301, p_D(2)=0.176, p_D(3)=0.125, p_D(4)=0.097, p_D(5)=0.079, p_D(6)=0.067, p_D(7)=0.058, p_D(8)=0.051, p_D(9)=0.046$\\
Quanto vale P(...)?
\begin{itemize}
    \item $0.222$
    \item Nessuna delle altre risposte
    \item $0.067$
    \item $0.155$
    \item $0.051$
\end{itemize}

\textbf{Esercizio 16} \underline{Nota}: alcune parti del testo indecifrabili\\
Durante un sondaggio, ad un gruppo di ragazzi è stato chiesto quale superpotere avrebbero voluto avere. Le risposte sono sintetizzate nella seguente tabella
\begin{center}
    \begin{tabular}{c|cc|c}
        & Maschi & Femmine & Totale \\ \hline
        Saper volare & 30 & 10 & 40 \\
        Invisibilità & 12 & 32 & 44 \\
        Altro & 10 & 6 & 16 \\ \hline
        Totale & 52 & 48 & 100 \\
    \end{tabular}
\end{center}
Si scelga a caso un ragazzo di questo gruppo e si considerino gli eventi $A$="Il ragazzo scelto è maschio" e $B$="il ragazzo scelto vorrebbe saper volare". Allora $P(A|B)$ vale
\begin{itemize}
    \item $\frac{3}{10}$
    \item Nessuna delle altre risposte
    \item $\frac{15}{26}$
    \item $\frac{3}{4}$ (Risposta Corretta)
    \item $\frac{10}{13}$
\end{itemize}

\textbf{Esercizio 17}
Benji si allena parando i rigori. Ad ogni tiro, indipendentemente dall'altro, ha una probabilità di 0.8 di parare. Qual è la probabilità che pari tutti i prossimi 3 tiri?
\begin{itemize}
    \item $0.512$ (Risposta Corretta)
    \item Nessuna delle altre risposte
    \item $0.6$
    \item $1$
    \item $0.008$
\end{itemize}

\textbf{Esercizio 18}
Siano $A, B$ eventi tali che $P(A)=0.2, P(B)=0.5$ e $P(A|B)=0.4$. Quanto vale la probabilità dell'evento $A\cup B$?
\begin{itemize}
    \item $0.7$
    \item $0.78$
    \item Nessuna delle altre risposte (Risposta Corretta)
    \item $0.62$
    \item $0.24$
\end{itemize}

\textbf{Esercizio 19}
Lancio due dadi. Qual è la probabilità che la somma dei due punteggia sia 4?
\begin{itemize}
    \item $\frac{1}{18}$
    \item $\frac{1}{6}$
    \item $\frac{1}{36}$
    \item $\frac{1}{12}$ (Risposta Corretta)
    \item Nessuna delle altre risposte
\end{itemize}

\textbf{Esercizio 20} \underline{Nota}: alcune parti del testo indecifrabili\\
Mia mamma non era una gran cuoca ... 5 formati di pasta (spaghetti, penne, farfalle, fettuccine, fusilli) e 3 condimenti (pomodoro, ragù, carbonara), che abbinava a caso. Qual era la probabilità di non mangiare fusilli al ragù?
\begin{itemize}
    \item Nessuna delle altre risposte
    \item $\frac{2}{3}$
    \item $\frac{4}{5}$
    \item $\frac{14}{15}$ (Risposta Corretta)
    \item $\frac{1}{15}$
\end{itemize}

\textbf{Esercizio 21} \underline{Nota}: alcune parti del testo indecifrabili\\
Le lettere della parola STATISTICA vengono ... e mescolate. Un ... viene pescato a caso. Qual è la probabilità di pescare una consonante?
\begin{itemize}
    \item $\frac{1}{5}$
    \item $\frac{5}{11}$
    \item $\frac{2}{5}$
    \item Nessuna delle altre risposte (Risposta Corretta)
    \item $\frac{6}{11}$
\end{itemize}

\textbf{Esercizio 22} \underline{Nota}: alcune parti del testo indecifrabili\\
Da un mazzo di carte da Poker (52 carte) vengono pescate tre carte \textit{senza reinserimento}. Si considerino gli eventi $A$="le prime due carte sono entrambe di cuori" e $B$="la prima e la terza carta non sono dello stesso seme". Allora
\begin{itemize}
    \item $A\cap B^c=$"le tre carte sono di cuori" (Risposta Corretta)
    \item $A\cap B$ è l'evento impossibile 
    \item Nessuna delle altre risposte
    \item $A\cup B=$"le prime due carte sono di cuori e la terza non è di cuori"
    \item $A^c\cap B=$"la seconda e la terza carta non sono entrambe di cuori"
\end{itemize}

\textbf{Esercizio 23} \underline{Nota}: alcune parti del testo indecifrabili\\
Siano $A, B$ eventi tali che $P(A)=0.2, P(B)=0.3$. Qual è la probabilità dell'evento $A\cup B$?
\begin{itemize}
    \item $0.5$
    \item $0.66$
    \item $0.44$
    \item $0.6$
    \item Nessuna delle altre risposte  (Risposta Corretta)
\end{itemize}

\textbf{Esercizio 24} \underline{Nota}: alcune parti del testo indecifrabili\\
Impresa d'appalti e distribuzione $X$\\

\textbf{Esercizio 25}
Siano $A, B, C$ eventi \textit{indipendenti} tali che $P(A)=P(B)=P(C)=\frac{1}{2}$. Qual è la probabilità dell'evento $(A\cap B)\cup C$?
\begin{itemize}
    \item Non la posso calcolare
    \item $7/8$
    \item $5/8$ (Risposta corretta)
    \item $1/8$
    \item $3/8$
\end{itemize}

\textbf{Esercizio 26}
Sia $X$ una variabile aleatoria discreta tale che $P(X=0)=P(X=1)=\frac{1}{4}$ e $P(X=-2)=\frac{1}{2}$. Definiamo $Y=X-X^2$. Allora l'alfabeto di $Y$ è
\begin{itemize}
    \item ${0, -2, -6}$ 
    \item ${0, -2}$
    \item ${0, -2, 2}$
    \item ${0, -6}$
    \item Nessuna delle altre risposte
\end{itemize}

\textbf{Esercizio 27}
Lanciamo un dado. Consideriamo i seguenti eventi: $E_> = $"esce un punteggio maggiore di 4", $E_i =$ "esce il punteggio $i$" (con $i = 1,..., 6)$ e $E =$ "esce un punteggio divisibile per 3". Quale delle seguenti affermazioni è \textit{falsa}?
\begin{itemize}
    \item Gli eventi $E_5$ ed $E$ sono incompatibili
    \item Nessuna delle altre risposte (Risposta corretta)
    \item Gli eventi $E_1$, $E_2$, $E_3$ ed $E_>$ non sono una partizione
    \item Gli eventi $E_i$ con $i = 1,...,6$ sono una partizione
    \item Gli eventi $E_4$, $E_2$, $E_1$, $E$ ed $E_>$ non sono una partizione
\end{itemize}

\section*{Anno 20/21}
Le risposte degli esercizi di questa sezione sono prese dalle soluzioni dei questionari, quindi sicuramente corrette.

\subsection*{Primo questionario}
\textbf{Argomenti}\\
Spazi campionari, disposizioni e combinazioni, probabilità condizionate, eventi indipendenti\\

\textbf{Esercizio 1}
Siano $A, B$ eventi \textit{indipendenti} tali che $P(A) = 0.2$ e $P(B) = 0.3$. Qual è la probabilità dell'evento $A \cup B$?
\begin{itemize}
    \item Non la posso calcolare
    \item $0.5$
    \item $0.44$ (Risposta Corretta)
    \item $0.66$
    \item $0.6$
\end{itemize}

\textbf{Esercizio 2}
Siano $A,B$ eventi tali che $P(A|B) = P(B) = \frac{1}{2}$. Qual è la probabilità dell'evento $(A^c \cap B^c)$?
\begin{itemize}
    \item $\frac{1}{4}$
    \item $\frac{1}{2}$
    \item $\frac{3}{8}$ 
    \item Non la posso calcolare
    \item $\frac{3}{4}$ (Risposta Corretta)
\end{itemize}

\textbf{Esercizio 3}
Da un mazzo di carte da Poker (52 carte) si estraggono, in successione e senza reinserimento, tre carte. Si considerino gli eventi $A=$"la prima e la terza carta non sono dello stesso seme" e $B=$"le ultime due carte sono entrambe di fiori". Allora
\begin{itemize}
    \item $A\cap B$ è l'evento impossibile
    \item $A^c\cap B=$"le tre carte sono di fiori" (Risposta Corretta)
    \item Nessuna delle altre risposte
    \item $A\cap B^c=$"la prima e la seconda carta non sono entrambe di fiori"
    \item $A\cup B=$"la prima carta non è di fiori e le due ultime sono di fiori"
\end{itemize}

\textbf{Esercizio 4}
Estraggo una carta da un mazzo di carte da Poker (52 carte). Qual è la probabilità di scegliere una carta di quadri sapendo che la carta scelta è rossa?
\begin{itemize}
    \item $\frac{1}{4}$
    \item $\frac{1}{13}$
    \item $\frac{1}{26}$
    \item Nessuna delle altre risposte
    \item $\frac{1}{2}$ (Risposta Corretta)
\end{itemize}

\textbf{Esercizio 5}
Andrea, Stefano, Luca e Olga hanno formato un quartetto, con quattro diversi strumenti. Se ognuno di essi può suonare tutti e quattro gli strumenti, quanti sono i modi possibili per formare il quartetto?
\begin{itemize}
    \item Nessuna delle altre risposte
    \item $18$
    \item $24$ (Risposta Corretta)
    \item $36$
    \item $20$
\end{itemize}

\textbf{Esercizio 6}
Siano $A, B$ eventi tali che $P(A) = 0.2, P(B) = 0.5 e P(B|A) = 0.4$. Quanto vale la probabilità dell'evento $A\cup B$?
\begin{itemize}
    \item $0.24$
    \item $0.7$
    \item $0.62$ (Risposta Corretta)
    \item $0.78$
    \item Nessuna delle altre risposte
\end{itemize}

\textbf{Esercizio 7}
Supponiamo di avere due mazzi di carte da Poker (52 carte) e di pescare una carta da ciascun mazzo. Qual è la probabilità che entrambe le carte siano di quadri?
\begin{itemize}
    \item $\frac{3}{51}$
    \item $\frac{3}{104}$
    \item $\frac{1}{16}$ (Risposta Corretta)
    \item Nessuna delle altre risposte
    \item $\frac{5}{84}$
\end{itemize}

\textbf{Esercizio 8}
Il mio amico Hugo fa il cameriere di sala in un famoso ristorante parigino. Ogni sera serve una decina di tavoli. La probabilità che ad un tavolo chiedano un seggiolone è 0.03. Qual è la probabilità (approssimata) che durante una serata sia necessario almeno un seggiolone?
\begin{itemize}
    \item $0.26$ (Risposta Corretta)
    \item $0.03$
    \item $0.74$
    \item Nessuna delle altre risposte
    \item $0.97$
\end{itemize}

\textbf{Esercizio 9}
Siano $A,B$ eventi tali che $P(A|B^c)=P(B)=\frac{1}{2}$. Qual è la probabilità dell'evento $A^c\cup B$?
\begin{itemize}
    \item $\frac{3}{8}$
    \item $\frac{1}{4}$
    \item $\frac{1}{2}$
    \item Non la posso calcolare
    \item $\frac{3}{4}$ (Risposta Corretta)
\end{itemize}

\textbf{Esercizio 10}
Lanciamo contemporaneamente un dado e una moneta. Qual è la probabilità che il dado dia 6 e la moneta testa?
\begin{itemize}
    \item $\frac{1}{2}$
    \item $\frac{1}{12}$ (Risposta Corretta)
    \item Nessuna delle altre risposte
    \item $\frac{2}{3}$
    \item $\frac{1}{6}$
\end{itemize}

\textbf{Esercizio 11}
In una classe di 8 alunni, 3 hanno fatto i compiti. Se l'insegnante ne sceglie 2 a caso, qual è la probabilità che entrambi abbiano svolto i compiti?
\begin{itemize}
    \item $\frac{3}{8}$
    \item $\frac{3}{56}$
    \item $\frac{3}{28}$ (Risposta Corretta)
    \item $\frac{3}{14}$
    \item Nessuna delle altre risposte
\end{itemize}

\textbf{Esercizio 12}
Siano $A, B$ eventi tali che $P(A)=0.5, P(B)=0.8$ e $P(A\cap B)=0.2$. Quanto vale $P(A|B)$?
\begin{itemize}
    \item $0.4$
    \item $0.65$
    \item $0.25$ (Risposta Corretta)
    \item $0.5$
    \item Nessuna delle altre risposte
\end{itemize}

\textbf{Esercizio 13}
La probabilità che un volo parta in orario è $0.8$, la probabilità che arrivi in orario è $0.4$ e la probabilità che parta e arrivi in orario è $0.2$. Quanto vale la probabilità che un aereo arrivi in orario, dato che è partito in orario?
\begin{itemize}
    \item $0.2$
    \item $0.5$
    \item $0.75$
    \item $0.25$ (Risposta Corretta)
    \item Nessuna delle altre risposte
\end{itemize}

\textbf{Esercizio 14}
Da un mazzo di carte da Poker (52 carte) si estrae una carta. Qual è la probabilità che sia la regina di fiori o il re di quadri?
\begin{itemize}
    \item $\frac{1}{52}$
    \item Nessuna delle altre risposte
    \item $0$
    \item $\frac{1}{26}$ (Risposta Corretta)
    \item $\frac{1}{2704}$
\end{itemize}

\textbf{Esercizio 15}
Estraggo una carta da un mazzo di carte da Poker (52 carte). Considero gli eventi $A=$"estraggo un asso" e $B=$"estraggo una carta di fiori". Quanto vale $P(A|B)$?
\begin{itemize}
    \item $\frac{1}{52}$
    \item $\frac{1}{4}$
    \item $\frac{1}{13}$ (Risposta Corretta)
    \item $\frac{17}{52}$
\end{itemize}

\textbf{Esercizio 16}
Si considerino lo spazio campionario $\Omega =\{Do, Re, Mi, Fa, Sol, La, Si\}$ e gli eventi $A=\{Do, Re, Si\}, B=\{Re, Mi, Fa\}$ e $C=\{La\}$. Quali delle seguenti terne è una terna di eventi a 2 a 2 disgiunti?
\begin{itemize}
    \item $\{A\cup B, C, (A\cup B \cup C)^c\}$ (Risposta Corretta)
    \item $\{A \\ B, B, C^c\}$
    \item $\{A\cap B, B^c, C\}$
    \item Nessuna delle altre risposte
    \item $\{A, B, C\}$
\end{itemize}

\textbf{Esercizio 17}
Le lettere della parola STATISTICA vengono scritte su dei bigliettini che vengono poi messi in un sacchetto e mescolati. Un bigliettino viene quindi pescato a caso. Qual è la probabilità di pescare una vocale?
\begin{itemize}
    \item $\frac{5}{11}$
    \item $\frac{1}{5}$
    \item Nessuna delle altre risposte
    \item $\frac{2}{5}$ (Risposta Corretta)
    \item $\frac{6}{11}$
\end{itemize}

\textbf{Esercizio 18}
Al panificio "La dolce baguette" fanno dei dolci squisiti, a cui è davvero difficile resistere. Ogni persona che entra per prendere il pane, indipendentemente dalle altre, ne è così tentata che con una probabilità $0.85$ aggiungerà un dolce alla sua spesa. Stamattina il panificio ha avuto $7$ clienti. Qual è la probabilità (approssimata) che almeno uno non abbia comperato un dolce?
\begin{itemize}
    \item $0.32$
    \item $0.15$
    \item $0.68$ (Risposta Corretta)
    \item Nessuna delle altre risposte
    \item $\frac{1}{7}$
\end{itemize}

\textbf{Esercizio 19}
Gli studenti di un corso di laurea vengono classificati sulla base di due caratteristiche: il sesso e il tipo di scuola superiore di provenienza. I dati sono riportati nella seguente tabella
\begin{center}
    \begin{tabular}{c|cc|c}
        & LICEO & ALTRA SCUOLA SUPERIORE & Totale \\ \hline
        MASCHI & $47$ & $63$ & $110$ \\
        FEMMINE & $62$ & $51$ & $113$ \\ \hline
        Totale & $109$ & $114$ & $223$ \\
    \end{tabular}
\end{center}
Si scelga a caso uno studente di questo corso di laurea e si considerino gli eventi: $A=$"lo studente scelto è maschio" e $B=$"lo studente scelto proviene da un liceo". Allora $P(A|B)$ vale
\begin{itemize}
    \item $\frac{47}{98}$
    \item $\frac{47}{110}$
    \item Nessuna delle altre risposte
    \item $\frac{47}{223}$
    \item $\frac{47}{109}$ (Risposta Corretta)
\end{itemize}

\textbf{Esercizio 20}
Lanciamo un dado. Consideriamo i seguenti eventi: $E_<=$"esce un punteggio minore di 3", $E_i=$"esce il punteggio $i$" (con $i=1,...6$) e $E=$"esce un punteggio dispari". Quali delle seguenti affermazioni è vera?
\begin{itemize}
    \item Gli eventi $E_2, E_6$ ed $E$ sono una partizione
    \item Nessuna delle altre risposte
    \item Gli eventi $E_i$, con $i=1,...,6$ sono una partizione (Risposta corretta)
    \item Gli eventi $E_4, E_6, E$ ed $E_<$ sono una partizione
    \item Gli eventi $E_4$ ed $E_<$ non sono incompatibili
\end{itemize}

\textbf{Esercizio 21}
Il veliero dell'audace Capitano Kirk è a poche miglia dal galeone del famigerato pirata Gambadilegno. Il capitano ha probabilità $\frac{3}{5}$ di colpire la nave del pirata coi suoi cannoni; mentre il pirata, che ha solo un occhio buono, colpisce la nave del capitano con probabilità $\frac{1}{5}$. Se entrambi danno fuoco alle polveri nello stesso momento, qual è la probabilità che il capitano Kirk colpisca la nave pirata e Gambadilegno invece manchi la nave del capitano?
\begin{itemize}
    \item $\frac{4}{5}$
    \item $\frac{12}{25}$ (Risposta Corretta)
    \item $\frac{2}{5}$
    \item $\frac{4}{25}$
    \item Nessuna delle altre risposte
\end{itemize}

\textbf{Esercizio 22}
Siano $A, B$ eventi tali che $P(A\cap B)=0.6, P(A^c)=0.3$ e $P(B)=0.8$. Qual è la probabilità dell'evento $A\cup B$?
\begin{itemize}
    \item $0.9$ (Risposta Corretta)
    \item $0.5$
    \item Non la posso calcolare
    \item $0.4$
    \item $0.8$
\end{itemize}

\textbf{Esercizio 23}
Benji si allena parando i rigori. Ad ogni tiro, indipendentemente dagli altri, ha la probabilità $0.8$ di parare. Qual è la probabilità che sbagli tutte le prossime $3$ parate?
\begin{itemize}
    \item $0.008$ (Risposta Corretta)
    \item $0.6$
    \item $1$
    \item Nessuna delle altre risposte
    \item $0.512$
\end{itemize}

\textbf{Esercizio 24}
Gli studenti di un corso di laurea vengono classificati sulla base di due caratteristiche: il sesso e il tipo di scuola superiore di provenienza. I dati sono riportati nella seguente tabella
\begin{center}
    \begin{tabular}{c|cc|c}
        & LICEO & ALTRA SCUOLA SUPERIORE & Totale \\ \hline
        MASCHI & $47$ & $63$ & $110$ \\
        FEMMINE & $62$ & $51$ & $113$ \\ \hline
        Totale & $109$ & $114$ & $223$ \\
    \end{tabular}
\end{center}
Si scelga a caso uno studente di questo corso di laurea e si considerino gli eventi: $A=$"lo studente scelto è femmina" e $B=$"lo studente scelto proviene da un liceo". Allora $P(B|A)$ vale
\begin{itemize}
    \item Nessuna delle altre risposte
    \item $\frac{62}{223}$
    \item $\frac{62}{113}$ (Risposta Corretta)
    \item $\frac{62}{125}$
    \item $\frac{62}{109}$
\end{itemize}

\textbf{Esercizio 25}
Stai giocando ad un gioco in cui devi difendere un villaggio da un'invasione di orchi. Ci sono $3$ personaggi (elfo, hobbit, uomo) e $5$ strumenti di difesa (magia, spada, scudo, fionda, ombrello). Se scegli a caso il tuo personaggio e la tua arma, con quale probabilità sarai un umano?
\begin{itemize}
    \item $\frac{1}{5}$
    \item $\frac{1}{15}$
    \item $\frac{8}{15}$
    \item Nessuna delle altre risposte
    \item $\frac{1}{3}$
\end{itemize}

\textbf{Esercizio 26}
Siano $A, B$ eventi tali che $P(A\cap B^c)=0.3$. Qual è la probabilità dell'evento $A^c\cap B$?
\begin{itemize}
    \item $0.3$
    \item Non la posso calcolare (Risposta Corretta)
    \item $0.6$
    \item $0.4$
    \item $0.7$
\end{itemize}

\textbf{Esercizio 27}
La mia amica Martina adora lo yoga e segue un corso avanzato tre volte alla settimana. Il $60\%$ dei partecipanti a questo corso sono donne; il $10\%$ dei partecipanti sono donne con un abbigliamento sui toni del viola. Sapendo che si è scelta una donna della classe, qual è la probabilità che indossi qualcosa di viola?
\begin{itemize}
    \item $\frac{1}{10}$
    \item $\frac{7}{10}$
    \item Nessuna delle altre risposte
    \item $\frac{3}{5}$
    \item $\frac{1}{6}$ (Risposta Corretta)
\end{itemize}

\textbf{Esercizio 28}
Siano $A, B$ eventi \textit{indipendenti} tali che $P(A^c)=0.1$ e $P(B)=0.7$. Qual è la probabilità dell'evento $A^c\cap B$?
\begin{itemize}
    \item Non la posso calcolare
    \item $0.27$
    \item $0.07$ (Risposta Corretta)
    \item $0.03$
    \item $0.63$
\end{itemize}

\textbf{Esercizio 29}
Siano $A, B$ eventi tali che $P(A^c\cup B)=0.4$. Qual è la probabilità dell'evento $A\cup B^c$?
\begin{itemize}
    \item $0.7$
    \item $0.6$
    \item $0.3$
    \item $0.4$
    \item Non la posso calcolare (Risposta Corretta)
\end{itemize}

\textbf{Esercizio 30}
Quanti sono gli anagrammi della parola PROBABILITÀ (fare come se l'accento non ci fosse)?
\begin{itemize}
    \item $39916800$
    \item $75600$
    \item $4989600$ (Risposta Corretta)
    \item $3628800$
    \item Nessuna delle altre risposte
\end{itemize}

\textbf{Esercizio 31}
Estraggo una carta da un mazzo di carte da Poker (52 carte). Considero gli eventi $A=$"estraggo una regina" e $B=$"estraggo una carta di cuori". Quanto vale $P(A|B)$?
\begin{itemize}
    \item $\frac{1}{13}$ (Risposta Corretta)
    \item $\frac{1}{4}$
    \item $\frac{1}{52}$
    \item $\frac{17}{52}$
\end{itemize}

\textbf{Esercizio 32}
Due dadi a sei facce sono così costruiti: il dado A ha due facce rosse e quattro facce blu; mentre il dado B ha tre facce rosse e tre facce blu. Se lancio i due dadi, qual è la probabilità che le facce siano dello stesso colore?
\begin{itemize}
    \item $\frac{1}{2}$ (Risposta Corretta)
    \item $\frac{1}{3}$
    \item Nessuna delle altre risposte
    \item $\frac{2}{3}$
    \item $\frac{1}{6}$
\end{itemize}

\textbf{Esercizio 33}
Siano $A, B$ eventi tali che $P(A|B)=\frac{1}{3}$. Quanto vale la probabilità condizionata $P(A^c|B)$?
\begin{itemize}
    \item Non la posso calcolare
    \item $\frac{1}{6}$
    \item $\frac{5}{6}$
    \item $\frac{1}{3}$
    \item $\frac{2}{3}$ (Risposta Corretta)
\end{itemize}

\textbf{Esercizio 34}
Mia nonna non era una gran cuoca, ma una volta al mese ci invitava tutti a pranzo da lei. Il suo repertorio di primi prevedeva $5$ formati di pasta (spaghetti, penne, farfalle, fettuccine, fusilli) e $3$ condimenti (pomodoro, ragù, carbonara), che abbinava a caso. Qual era la probabilità di non mangiare fusilli al ragù?
\begin{itemize}
    \item $\frac{2}{3}$
    \item Nessuna delle altre risposte
    \item $\frac{14}{15}$ (Risposta Corretta)
    \item $\frac{1}{15}$
    \item $\frac{4}{5}$
\end{itemize}

\textbf{Esercizio 35}
Durante un sondaggio, ad un gruppo di ragazzi è stato chiesto quale superpotere avrebbero voluto avere. Le risposte sono sintetizzate nella seguente tabella
\begin{center}
    \begin{tabular}{c|cc|c}
         & MASCHI & FEMMINE & Totale \\ \hline
        SAPER VOLAR & $30$ & $10$ & $40$ \\
        INVISIBILITÀ & $12$ & $32$ & $44$ \\
        ALTRO & $10$ & $6$ & $16$ \\ \hline
        Totale & $52$ & $48$ & $100$ \\
    \end{tabular}
\end{center}
Si scelga a caso un ragazzo di questo gruppo e si considerino gli eventi $A$="il ragazzo scelto è femmina" e $B$="il ragazzo scelto vorrebbe avere il dono dell'invisibilità". Allora $P(B|A)$ vale
\begin{itemize}
    \item $\frac{8}{11}$
    \item $\frac{5}{24}$
    \item $\frac{8}{25}$
    \item $\frac{2}{3}$ (Risposta Corretta)
    \item Nessuna delle altre risposte
\end{itemize}

\textbf{Esercizio 36}
Siano $A, B$ eventi \textit{indipendenti} tali che $P(A|B)=0.2$ e $P(B|A)=0.5$. Quanto vale la probabilità dell'evento $A\cup B$?
\begin{itemize}
    \item Non la posso calcolare 
    \item $0.6$ (Risposta Corretta)
    \item $o.4$
    \item $0.7$
    \item $0.8$
\end{itemize}

\textbf{Esercizio 37}
Un computer genera numeri casuali di $4$ cifre da $0000$ a $9999$ (estremi inclusi). Qual è la probabilità che il computer produca un numero che inizia e finisce con la cifra $1$?
\begin{itemize}
    \item $\frac{1}{100}$ (Risposta Corretta)
    \item $\frac{1}{10}$
    \item $\frac{1}{1000}$
    \item Nessuna delle altre risposte
    \item $\frac{1}{50}$
\end{itemize}

\textbf{Esercizio 38}
Nella mia scatola dei ricordi ho trovato un sacchetto di biglie di vetro Contiene ancora $21$ biglie: $8$ rosse, $7$ blu e $6$ verdi. Ne pesco una a caso. Qual è la probabilità che non sia né verde, né rossa?
\begin{itemize}
    \item $\frac{8}{21}$
    \item Nessuna delle altre risposte
    \item $\frac{1}{3}$ (Risposta Corretta)
    \item $\frac{5}{7}$
    \item $\frac{6}{21}$
\end{itemize}

\textbf{Esercizio 39}
Un'urna contiene $10$ palline rosse e $10$ palline viola. Si estraggono due palline \textit{senza reinserimento}. Qual è la probabilità che abbiano un colore diverso?
\begin{itemize}
    \item $\frac{10}{19}$ (Risposta Corretta)
    \item $\frac{1}{10}$
    \item $\frac{1}{2}$
    \item Nessuna delle altre risposte
    \item $\frac{2}{3}$
\end{itemize}

\textbf{Esercizio 40}
Siano $A, B$ eventi \textit{indipendenti} tali che $P(A)=\frac{1}{3}$ e $P(A\cap B^c)=\frac{1}{4}$. Qual è la probabilità dell'evento $B$?
\begin{itemize}
    \item $\frac{3}{4}$
    \item $\frac{2}{3}$
    \item Non la posso calcolare
    \item $\frac{1}{4}$ (Risposta esatta)
    \item $\frac{1}{2}$
\end{itemize}

\textbf{Esercizio 41}
Pesco una carta da un mazzo di carte da Poker (52 carte). Sapendo che è uscita una carta nera, qual è la probabilità che sia una figura?
\begin{itemize}
    \item $\frac{4}{13}$
    \item $\frac{6}{13}$
    \item $\frac{1}{2}$
    \item $\frac{3}{13}$ (Risposta Corretta)
    \item Nessuna delle altre risposte
\end{itemize}

\end{document}
\end{document}
